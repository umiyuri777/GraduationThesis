%% レイアウトに関する設定ファイル,必要に応じて修正してください

%% 以下,使用推奨パッケージ
%% タイトルフォント等の設定
\usepackage{bachelor_thesis}
%% 図の使用
\usepackage[dvipdfmx]{graphicx}

%% 以下は必要に応じて削除しても良い
%% 枠付き文章
\usepackage{ascmac}
%% 特殊文字のエラー回避
\usepackage{textcomp}
%% ソースコードの表示,日本語使用時の文字化け防止
\usepackage[svgnames]{xcolor}
\usepackage{listings,jvlisting,jlisting} 
\usepackage{jlisting}
%% 表のセル縦結合
\usepackage{multirow}
%% 注釈でURLを使う場合のエラー回避
\usepackage{url}
%%他,自由に追加してください
\usepackage{subcaption}
\usepackage{here}
\usepackage{amsmath}
\usepackage{array}
\usepackage{enumitem}
\usepackage{autobreak}
\usepackage{hyperref} 
\usepackage{pxjahyper}
\usepackage{cleveref} 

% 日本語フォーマットの定義
\crefformat{section}{#2第#1節#3}
\crefformat{subsection}{#2第#1節#3}
\crefformat{subsubsection}{#2第#1項#3} % ここを「小節」などに変えてもOK
\crefformat{figure}{#2図#1#3}
\crefformat{table}{#2表#1#3}
\crefformat{equation}{#2式(#1)#3}

\crefname{equation}{式}{式}% {環境名}{単数形}{複数形} \crefで引くときの表示
\crefname{figure}{図}{図}% {環境名}{単数形}{複数形} \crefで引くときの表示
\crefname{table}{表}{表}% {環境名}{単数形}{複数形} \crefで引くときの表示
\crefname{algorithm}{Algorithm}{Algorithm}

\crefname{section}{第}{第}
\creflabelformat{section}{#2#1節#3}
\crefname{subsection}{第}{第}
\creflabelformat{subsection}{#2#1節#3}

%% 余白設定,配布のものより狭くしています
\textheight=20.6truecm                % 高さ
\textwidth=14.5truecm                 % 横幅 (約36文字)
\oddsidemargin=0.6truecm              % 左の空きの幅
\evensidemargin=-3.8truecm            % 右の空きの幅

%% ソースコード表示の設定(参照: https://turgure.hatenablog.com/entry/2016/08/19/183501)
%% ソースコードの文字サイズをより小さくしたい場合は
%% \small部分を\footnotesizeや\scriptsizeにしてください
\lstset{
    frame=single,
    numbers=left,
    tabsize=2,
    columns=fixed,
    basewidth=0.5em,
    basicstyle=\ttfamily\small, 
}



%% 参考文献
\def\thebibliography#1{\section*{参考文献\markboth
 {参 考 文 献}{参 考 文 献}\addcontentsline{toc}{section}{参考文献}}\list
 {[\arabic{enumi}]}{\settowidth\labelwidth{[#1]}\leftmargin\labelwidth
 \advance\leftmargin\labelsep
 \usecounter{enumi}}
% \def\newblock{\hskip .11trueem plus .33trueem minus -.07trueem}
 \def\newblock{\hskip .11em plus .33em minus -.07em}
 \sloppy
 \sfcode`\.=1000\relax}
\let\endthebibliography=\endlist

%% 目次
\makeatletter%%
\renewcommand{\section}{%
\newpage
  \@startsection{section}% #1 見出し
   {1}% #2 見出しのレベル
   {\z@}% #3 横組みの場合,見出し左の空き(インデント量)
   {1.5\Cvs \@plus.5\Cdp \@minus.2\Cdp}% #4 見出し上の空き
   {.5\Cvs \@plus.3\Cdp}% #5 見出し下の空き (負の値なら見出し後の空き)
  {\raggedright\reset@font\large\bfseries}% 左揃え
}%
\makeatother%%
