% 本稿は立命館大学情報理工学部配布の卒論スタイルを利用し, \LaTeX でフォーマットした卒業論文のテンプレートです.
% 主に下記を追記しました.
% \begin{itemize}
% 	\item 目次の追加
% 	\item 余白の変更
% 	\item 本文のサンプル
% 	\item 図表,引用文献のサンプル
% 	\item 参考文献,謝辞の追加  
% \end{itemize}
% 次章より基本文法や図表の挿入について例を示します.
% 必要に応じて本texファイルのソースコードも参照ください.
% 本稿で説明している機能以外は各自でお調べください.

% 新しい章を挿入すると自動で改ページが行われます.

\section{序論}
近年,ビデオゲーム市場は世界規模で拡大の一途をたどっている.一般社団法人コンピュータエンターテインメント協会(CESA)が発行した『CESAゲーム産業レポート2025』によれば,世界市場規模は前年比5.0\%増の31兆円を超え,全世界で市場規模が拡大している\cite{CESA2025}.市場の成長に伴い,開発コストの増加や開発期間の長期化がビデオゲーム開発における深刻な問題点として挙げられている.英国の競争・市場庁(CMA)が2023年に公開した報告書によると,主要なAAAタイトル(大作ゲーム)の開発予算は2億ドルを超える規模に達しており,「Call of Duty」\footnote{https://www.callofduty.com/ja}のような一部のシリーズではすでに3億ドルを超える開発資金が投じられている\cite{CMA2023}.また,開発期間においても長期化の傾向は顕著であり,Unityの2024年ゲーミングレポートによれば,ビデオゲームの発売までの平均期間は,2022年の218日から2023年には304日へと,わずか1年で約40\%も増加している\cite{Unity2024}.

長期化した開発期間の中で,特に大きな割合を占めているのがテスト工程である\cite{ソフトウェア開発データ白書2018-2019}.したがって,開発期間の短縮とコスト削減を実現するためにはこのテスト工程を効率化することが必要であるが,その手法の一つとしてソフトウェアテストの自動化が挙げられる.ビデオゲーム開発においても,テスト自動化を試みた研究は複数存在する\cite{CEDIL2200, CEDIL2209}.しかし,ビデオゲーム開発においては「ゲームが楽しいこと」が最優先事項とされ,その芸術的な側面の強さゆえに,開発の過程で頻繁に要件や仕様が変更されるという特徴がある\cite{Kasurinen2014}.一部のゲーム企業ではスクリプトを用いた自動テスト手法\cite{MurphyHill2014}を採用しているが,要件や仕様の頻繁な変更に対して脆弱であり,変更のたびにテストスクリプトを作り直す必要がある.そのため,現状では自動テストを導入・維持するコストよりも,テスタを雇用して人手によるプレイテスティングを行った方が安価で済むケースが多く,自動化の普及を妨げる要因となっている\cite{MurphyHill2014}.特に,近年主流となっているオープンワールドと呼ばれる,プレイヤが自由にフィールド上を行動できるような形式のビデオゲームでは,テストの自動化はより困難とされる.主な要因として,オープンワールドゲームはプレイヤの自由度が高く,移動可能な範囲やインタラクションの組み合わせが膨大であるため,システムが取りうる状態数が膨大になり,網羅的な探索が難しくなることが挙げられる.これに対し,オープンワールドゲームのテスト自動化手法として,機械学習を用いた手法が提案されている\cite{Fan2022, Wang2023}.例えばMineDojo\cite{Fan2022}では,インターネット上の大規模な動画データと言語記述のペアから学習したMineCLIPという報酬モデルを導入することで,人間のプレイ動画に基づいた人間らしい挙動を学習・再現させ,膨大な状態空間を持つ環境においてもテストを行うことを可能にしている。しかし,この手法ではテストを実行するために大量の学習データを用意する必要があるため,開発段階で導入することは難しいと考えられる.

この課題に対し,オープンワールドゲームのテストにファジングを適用したBiFuzzという手法が提案されている\cite{Kato2024}.ファジング\cite{IPA_Fuzzing}はソフトウェアの脆弱性を検出するソフトウェアテスト手法の一つであり,セキュリティ対策の分野や自動運転システムなど幅広い分野で活用されている\cite{Guo2024}.ファジングは機械学習を用いた手法とは異なり,大量の学習データを必要としないため,開発期間内での導入が容易であるという利点がある.BiFuzzは,プレイヤの操作を入力として扱い,それを変異させることで効率的に不具合を探索するという方法を採用している.

一方で,ファジングには運用上の課題も存在する.ファジングは原理的に入力列を無限に変異させ生成し続けることが可能であるため,実行終了のタイミングを決定することが困難である.特にオープンワールドゲームのように探索空間が膨大な対象においては,単に実行時間を延長したとしても,必ずしも効率的に不具合を発見できるとは限らない.したがって,限られた開発リソースの中で最大限の効果を得るためには,実行結果を予測し,不具合を発見しやすいテストケースを優先的に生成する仕組みが必要である.BiFuzzでは,テストケースの生成に複数のパラメータが用いられる.しかし,現状におけるパラメータの決定はテスタの経験に委ねられており,パラメータの組み合わせと不具合発見率の関係性は明らかになっていない.もし,ファジングの実行結果を事前に予測し,不具合発見に寄与するパラメータを特定できれば,不具合発見の確率が高いテストケースを優先的に実行することが可能となる.その結果,限られた実行時間内であっても最大限の成果が得られ,テスト工程の大幅な効率化が期待できる.

そこで本研究では,BiFuzz におけるパラメータの組み合わせと不具合発見の関係性を調査し,最適なモデルを導出するためのデータ収集および分析プロセスを提案する.具体的には,ファジングの実行結果を収集し,ロジスティック回帰分析,決定木,ランダムフォレストといったアルゴリズムを用いて,「その設定で不具合を発見できるか」を予測するモデルの構築を試みる。本研究の有効性を検証するために,BiFuzzの研究で使用されていたビデオゲームを対象に実験を行い,提案手法によって不具合の発見を予測できるモデルが構築可能であるかを明らかにする。