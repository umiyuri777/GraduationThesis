\section{序論}
% 本稿は立命館大学情報理工学部配布の卒論スタイルを利用し, \LaTeX でフォーマットした卒業論文のテンプレートです.
% 主に下記を追記しました.
% \begin{itemize}
% 	\item 目次の追加
% 	\item 余白の変更
% 	\item 本文のサンプル
% 	\item 図表,引用文献のサンプル
% 	\item 参考文献,謝辞の追加  
% \end{itemize}
% 次章より基本文法や図表の挿入について例を示します.
% 必要に応じて本texファイルのソースコードも参照ください.
% 本稿で説明している機能以外は各自でお調べください.

% 新しい章を挿入すると自動で改ページが行われます.

近年,ビデオゲーム市場は世界規模で拡大の一途をたどっている.一般社団法人コンピュータエンターテインメント協会(CESA)が発行した『CESAゲーム産業レポート2025』によれば,世界市場規模は前年比5.0\%増の31兆円を超え,全世界で市場規模が成長していることが示されている.また,日本国内においてもビデオゲーム人口は約5,000万人と横ばいに推移しており,国民の娯楽の一つとして広く浸透している\cite{CESA2025}.この急速な市場の成長に伴い,開発コストの増加や開発期間の長期化がビデオゲーム開発における深刻な問題点として挙げられている.英国の競争・市場庁(CMA)が2023年に公開した報告書によると,主要なAAAタイトル(大作ゲーム)の開発予算は2億ドルを超える規模に達しており,「Call of Duty」\footnote{https://www.callofduty.com/ja}のような一部のシリーズではすでに3億ドルを超える開発資金が投じられている\cite{CMA2023}.また,開発期間においても長期化の傾向は顕著であり,Unityの2024年ゲーミングレポートによれば,ビデオゲームの発売までの平均期間は,2022年の218日から2023年には304日へと,わずか1年で約40\%も増加している\cite{Unity2024}.こうした開発期間の長期化において,特に大きな割合を占めているのがテスト工程である.ゲームの大規模化・複雑化に伴い,バグの発見と修正に要する時間は肥大化しており,開発プロセス全体における最大のボトルネックとなっている.したがって,開発期間の短縮とコスト削減を実現するためには,このテスト工程を効率化することが急務である.
このような開発期間の長期化を解消するための手法の一つとして,ソフトウェアテストの自動化が挙げられる.ビデオゲーム開発においても,テスト工程の効率化を目指して自動化を試みた研究は複数存在する\cite{CEDIL2200, CEDIL2209}.しかし,ビデオゲーム開発においては「ゲームが楽しいこと」が最優先事項とされ,その芸術的な側面の強さゆえに,開発の過程で頻繁に要件や仕様が変更されるという特徴がある\cite{Kasurinen2014}. 一部のゲーム企業ではスクリプトを用いた自動テスト手法を用いている\cite{MurphyHill2014}が,こうした頻繁な変更に対して脆弱であり,仕様が変わるたびにテストスクリプトを作り直す必要がある.そのため,現状では自動テストを導入・維持するコストよりも,テスタを雇って人手によるプレイテスティングを行った方が安価で済むケースが多く,自動化の普及を妨げる要因となっている\cite{MurphyHill2014}.特に,近年主流となっているオープンワールドというプレイヤが自由にフィールド上を行動できるような形式のビデオゲームでは,テストの自動化はより困難である.オープンワールドゲームはプレイヤの自由度が高く,移動可能な範囲やインタラクションの組み合わせが膨大であるため,システムが取りうる状態数が膨大になり,網羅的な探索が難しくなる.オープンワールドゲームのテスト自動化手法として,機械学習を用いた手法が提案されている\cite{Fan2022, Wang2023}.しかし,テストを実行するために大量の学習データを用意する必要があり,開発段階で導入することは難しいと考えられる.

この課題に対し,オープンワールドゲームのテストにファジングを適用した「BiFuzz」という手法が提案されている\cite{Kato2024}.ファジングはソフトウェアの脆弱性を検出するソフトウェアテスト手法の一つであり\cite{IPA_Fuzzing},セキュリティ対策の分野や自動運転システムなど幅広い分野で活用されている\cite{Guo2024}.ファジングは機械学習を用いた手法とは異なり,大量の学習データを必要としないため,開発期間内での導入が容易であるという利点がある.BiFuzzは,プレイヤの操作を入力として扱い,それを変異させることで効率的にバグを探索するアプローチをとっている.しかし,BiFuzzにも課題が存在する.BiFuzzの実行にはパラメータの設定が必要であり,その決定はテスタに委ねられている.パラメータの設定が不適切な場合,バグを発見しにくい非効率なテストを実行してしまう恐れがある. そこで本研究では,ファジングの実行結果を収集し,そのデータを用いて「不具合を発見できるか」を予測するモデルを構築するプロセスを提案する.予測モデルの構築には,ロジスティック回帰分析,決定木,ランダムフォレストといったアルゴリズムを使用する. 本研究の有効性を検証するために,BiFuzzの研究で使用されていたビデオゲームを対象に実験を行い,提案手法によって不具合の発見を予測できるモデルが構築可能であるかを明らかにする.