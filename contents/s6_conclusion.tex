\section{まとめ}
本研究では,オープンワールドゲームに対するファジングツールであるBiFuzzにおいて,パラメータの決定がテスタの経験に委ねられているという課題に対し,BiFuzzにおけるパラメータの組み合わせと不具合発見の関係性を調査し,最適なモデルを導出するためのデータ収集および分析プロセスを提案した.提案手法では,ファジングの実行データを収集し,ロジスティック回帰分析,決定木,ランダムフォレストといった回帰モデルを用いることで,不具合を発見できる可能性が高いパラメータ設定を特定することを目指した.

実験では,BiFuzz の研究で使用された「STAR COLLECTION」を対象に,提案手法の有効性を検証した.スタックする不具合を対象にデータを収集・分析した結果,決定木およびランダムフォレストを用いることで,ベースラインと比較して高いF値を達成し,有効な予測モデルが構築可能であることを確認した.また,不具合発見にはマップ上の障害物の密度といった環境要因よりも,経由地の数や生成範囲といった移動経路を決定するパラメータが支配的な影響を与えることを明らかにした.これにより,提案手法が効率的なパラメータ特定に寄与することを確認した.

本研究の課題として,対象としたゲームシステムの単純さが挙げられる.本実験で使用した「STAR COLLECTION」は,静的なオブジェクトだけで構成された単純なゲームシステムとなっている.実際のオープンワールドゲームでは,物理演算を用いるアクションや時間経過による環境変化などのより複雑な要素が含まれるため,今後はより複雑な環境下での有効性を検証する必要がある.以上を踏まえ,今後の展望として以下の3点を挙げる.

\begin{itemize}
    \item \textbf{より複雑なオープンワールドゲームへの適用}: 移動する床のような物理演算を伴うアクションや,時間経過によって環境が変化するマップなど,より複雑な要素を持つオープンワールドゲームに対して提案手法を適用し,有効性を調査する.
    \item \textbf{不具合発見までの時間の予測}: 本研究では不具合発見の成否を予測したが,テストのさらなる効率化のためには実行時間を想定することも重要である.そこで,回帰分析等を用いてファジング実行から不具合発見までにかかる時間を予測可能にし,限られた計算リソースの適切な配分を実現する.
    \item \textbf{適用プロセスの実証実験}: 本研究では10分割交差検証を用いてモデル構築プロセスの評価を行ったため,ゲームに変更を加えた上での検証を行うことができなかった.今後は,実際にゲームへ変更を加えた後に,提案手法における適用プロセスを再現した実験を行い,実用性を評価する.
\end{itemize}

本論文で得られた知見を活かし,オープンワールドゲームのテスト自動化において,より効率的かつ実用的な不具合予測手法の確立を目指す.