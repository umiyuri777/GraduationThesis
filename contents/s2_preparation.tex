% \section{準備}
% \subsection{関連技術}
% 基本的な \LaTeX の記法は以下になります.
% より詳細な文法については各自でお調べください.
% \subsubsection*{章,節,項}

% \begin{itemize}
% 	\item \verb|\section{章タイトル}|
% 	\item \verb|\subsection{節タイトル}|
% 	\item \verb|\subsubsection{項タイトル}|
% \end{itemize}

% また,\verb|\subsubsection*{タイトル}|と記述すると,見出しの番号が非表示になります.
% 例えば「2.1.1 章,節,項」が「章,節,項」のみの表示となります.


% \subsubsection*{改行}

% \begin{itemize}
% 	\item \verb|\\|
% 	\item \verb|\par|
% 	\item 空白行
% 	\item \verb|\linebreak|
% \end{itemize}

% タイトル内でも改行できるため,適切な位置で改行を挿入してください.
% なお,改行直後の段落では自動で字下げが行われます.
% 字下げしたくない場合は\verb|\noindent|を段落の直前に挿入してください(本texファイル参照).

% \begin{itembox}[l]{改行の例(ソースコード)}
% 立命館大学 \textbackslash\textbackslash \\
% Ritsumeikan University \textbackslash par \\
% 情報理工学部 \\
% \\
% College of Information Science and Engineering\\
% \\
% \\
% \\
% システムアーキテクトコース \textbackslash linebreak\\
% System Architect Course
% \end{itembox}

% \begin{itembox}[l]{改行の例(出力)}
% 立命館大学 \\
% Ritsumeikan University \par
% 情報理工学部 

% College of Information Science and Engineering



% システムアーキテクトコース \linebreak
% System Architect Course
% \end{itembox}


% \subsubsection*{特殊文字}
% \LaTeX は直接入力ができない文字があります.前項の\verb|\\|もコード中に記入したら改行されてしまい,バックスラッシュが表示されません.
% 他によく使うものだと矢印( $\rightarrow$ , $\leftarrow$ )や 波ダッシュ( $\sim$ ) があります.
% 右向きの矢印は\verb|$\rightarrow$|と入力すると表示されます.
% 波ダッシュは\verb|$\sim$|と記述するか,\verb|\usepackage{textcomp}|などを使うことで記述できます.

% 他,数学記号やギリシャ文字,引用文献の人名におけるウムラウト( \"{a} )やアクセント( \'{o} )なども通常の入力ではエラーが生じる場合があります.
% 必要に応じてコマンドを使用してください.

% \subsubsection*{図}
% 図を挿入する際は,\verb|figure|を用います.
% 図は特に挿入位置を指定する必要は無く,自動で最適な箇所に挿入されます.

% 下記に例を示します.
% 下記は図を挿入しつつ,\verb|\begin{figure}~\end{figure}|で囲むことで図を中央揃えに配置しています.
% 例のように記述した場合,図\ref{fig:samplefig}が表示されます.

% \begin{lstlisting}[language=Tex]
% \begin{figure}[t!]
%     \begin{center}
%        \includegraphic[\linewide]{img/sample.png}
%        \caption{図のサンプル}
%     \end{center}
%     \label{fig:samplefig}
% \end{figure}
% \end{lstlisting}

% \verb|\begin{figure}[t!]|の\verb|[t!]|部分は図の挿入位置です.ソースコードの挿入位置(h),上部(t),下部(b)などがあります.
% 本稿ではレイアウトのため,挿入位置の後に\verb|!|を挿入しています(削除すると図が全て論文末尾に移動します).

% \verb|option|では画像の横幅(width)縦幅(height),倍率(scale)などを指定できます.
% \verb|\linewidth|はページの横幅と同様の値を指します.

% \verb|filename|で挿入する画像のファイル名を指定し,\verb|caption|にキャプションを記入します.
% ファイル名は相対パスで表記してください.

% \begin{figure}[b!]
%     \begin{center}
%         \includegraphics[width=\textwidth]{img/sample.png}
%         \caption{図のサンプル}
%         \label{fig:samplefig}
%     \end{center}
% \end{figure}


% 例では追加で,\verb|\caption{図のサンプル}|と\verb|\label{fig:samplefig}|を用いています.
% \verb|\caption{…}|でキャプションの指定,
% \verb|\label{…}|の使い方は\ref{sec:refexp}項で行います.

% \subsubsection*{表}
% 表の挿入は\verb|table|を用います.
% 下記の例の場合だと表\ref{table:sampletab}が出力されます.
% \begin{lstlisting}[language=Tex]
% \begin{table}[t]
%  \caption{表のサンプル}
%  \begin{center}     
%   \begin{tabular}{c|lll}
%    \hline
%    ID & キャンパス名 & 略称 & 大学  \\ \hline
%    1 & びわこ・くさつキャンパス & BKC & \multirow{2}{*}{立命館大学}\\
%    2 & 大阪・いばらきキャンパス & BKC & \\ 
%    3 & 豊中キャンパス &  & 大阪大学\\
%    \hline
%   \end{tabular}
%  \end{center}
%  \label{table:sampletab}
% \end{table}

% \end{lstlisting}

% \begin{table}[t]
%  \caption{表のサンプル}
%  \label{table:sampletab}
%  \begin{center}     
%   \begin{tabular}{c|l|l|l}
%    \hline
%    ID & キャンパス名 & 略称 & 大学  \\ \hline
%    1 & びわこ・くさつキャンパス & BKC & \multirow{2}{*}{立命館大学}\\ \
%    2 & 大阪・いばらきキャンパス & OIC & \\  \hline
%    3 & \multicolumn{2}{c|}{豊中キャンパス}  & 大阪大学\\
%    \hline
%   \end{tabular}
%  \end{center}
% \end{table}

% \verb|\caption|や\verb|\label|の使い方は\verb|figure|と同じです.

% \subsubsection*{引用文献,注釈}
% 文献の引用には\textbf{bibtex}を用いています.本論文の場合,texファイルに直接引用文献情報を書かず,\verb|reference.bib|というファイルに引用文献を記載しています.
% 引用方法は\ref{sec:ref}項を参照してください.
% bibファイルの一部を以下に示します.

% \begin{lstlisting}[language=Tex]
% @inproceedings{yoon2012,
%     author={YoungSeok Yoon and Brad A. Myers},
%     title={An Exploratory Study of Backtracking Strategies 
%                                                  Used by Developers},
%     booktitle={Proceedings of the 5th International Workshop on 
%                Cooperative and Human Aspects of Software Engineering},
%     year={2012},
%     month={June},
%     pages={138--144}
% }
% \end{lstlisting}

% 引用する対象が会議,ワークショップ等で発表された論文である場合\linebreak
% \verb|@inproceedings|で始め,論文誌の場合\verb|@article|で始めます.
% その後,例でいう\verb|yoon2012|部分は該当の文献を引用する時の名前になります.
% \verb|author|は著者一覧,\verb|title|は論文タイトル,\verb|booktitle|は発表された会議名(正確には該当論文が収録された予稿集名),\verb|year, month|は発表された年月,\verb|page|はページ番号です.
% 論文誌の場合\verb|booktitle|でなく\verb|journal|になります.より詳細な情報はbibtexの書き方など\footnote{https://mathlandscape.com/latex-bib/ など}で調べて下さい.

% 文献以外で説明の補足を行う場合は注釈機能を用います.
% 注釈とはこのように\footnote{https://卒業論文.com/2020/04/16/post-250/}該当のページ内下部に表示されるため,論文を読むときの補足に使われます.
% 記述は,
% スラッシュやチルダ(\verb|〜|)の文字化け回避のため,\verb|setting.tex|に\verb|\usepackage{url}|と記述しています.

% \subsection{関連技術} \label{sec:related}
% Texファイル内の参照を\ref{sec:refexp}項で,参考文献の引用を\ref{sec:ref}項で行います.
% \subsubsection{Tex内の参照} \label{sec:refexp}
% 特定の章や図表などを参照する際,数値を直接入力すると,章や図表を新しく挿入した際,参照番号がずれてしまう恐れがあります.
% そのため,Latexでは\textbf{相互参照}の機能があります.

% 相互参照の流れは,(1)参照したい章や図表に\verb|label|を付与し,
% (2)そのラベルを\verb|ref|コマンドで参照します.

% 例えば,\ref{sec:related}節から\ref{sec:refexp}項までのソースコードは下記のようになっています(わかりやすいように一部修正しています).

% \begin{lstlisting}[language=Tex]
% \subsection{関連技術} 
% Texファイル内の参照を\ref{sec:refexp}項で,
%             参考文献の引用を\ref{sec:ref}項で行います.

% \subsubsection{Tex内の参照} \label{sec:refexp}
% 特定の章や図表などを参照する際,数値を直接入力すると…(略)
% \end{lstlisting}

% 上記をコンパイルすると,下記のように出力されます.

% \begin{lstlisting}[language=Tex]
% 2.2 関連技術
%  Texファイル内の参照を2.2.1項で,参考文献の引用を2.2.2項で行います.

% 2.2.1 Tex内の参照
%  特定の章や図表などを参照する際,数値を直接入力すると…(略)
% \end{lstlisting}

% 相互参照を利用することで,例えば2.2.1項に新しい章が挿入され,現在の2.2.1項が2.2.2項や2.3.1項などに変化したとしても,\verb|``Texファイル内の参照を\ref{sec:refexp}項で''|の部分は常に
% \verb|\label{sec:refexp}|が付与された章を示します.

% 表や図のサンプルでもlabelとrefを利用しているので,本ソースコードをご確認ください.

% \subsubsection{文献の引用} \label{sec:ref}
% 引用のテスト\cite{IT人材白書2020}.
% これ\cite{yoon2012}もこれ\cite{sheil1983}もこれ\cite{sandeep2014}もテスト.
% 複数まとめて記述することもできます\cite{IT人材白書2020, yoon2012,sheil1983,sandeep2014}.
% 現状は引用順に参考文献が並びます.
% アルファベット順に引用文献を並べたいときは,本文末尾付近の\verb|\bibliographystyle{junsrt}|を\verb|\bibliographystyle{jplain}|に変更してください.

% コンパイル時に\verb|[??]|と表示されても,複数回コンパイルを行うと解決されることがあります.

\section{関連技術}
\subsection{オープンワールドゲーム}
オープンワールドゲームとは、広大かつ連続的な探索可能領域を持つビデオゲームのジャンルのことで,広大なマップ上をプレイヤが自由に動き回ることができるという特徴\cite{sca_openworld}がある.個別の場面ごとにマップが分断されることなくシームレスに接続されており,移動にロードを挟まない設計になっている.また,ゲームの進行順序が固定されている従来のゲーム形式と異なり,プレイヤは目的地へ到達するための経路選択や、ゲーム内で提示されるタスクやクエストを消化する順序を自由に決定できる\cite{sca_openworld}.

\subsection{ファジング}
ファジングとは,ソフトウェアテスト手法の一つである\cite{ipa_fuzzing_guide}.ファジングでは初期入力列を繰り返し変異させ,ファズという不具合を引き起こしやすい入力列を生成する.ビデオゲームにおけるファズとは,ゲームを開始した初期状態から不具合を引き起こす状態へ遷移する入力列のことを指す.

\subsection{オープンワールドゲームに対するファジングの適用}
オープンワールドゲームに対するファジングの適用事例として,Katoらによって提案された BiFuzz\cite{Kato2024}が挙げられる.BiFuzzは,オープンワールドゲーム特有の課題である広大な状態空間に対応するため,大域的ファジングと局所的ファジングという2段階でのファジングを行うことが特徴である.
前節で述べた通り,オープンワールドゲームは自由度が高いため,単純なランダム入力では状態空間が爆発的に増大し,網羅的なテストを行うことが困難である.これに対しBiFuzzでは,2段階でのファジングを組み合わせることで,人間らしい動作をする入力列を生成する.このように,プレイヤとしてあり得る挙動に絞って探索を行うことで実質的な状態空間を限定し,従来は困難であったオープンワールドゲームに対するファジングの実行を可能にしている.

