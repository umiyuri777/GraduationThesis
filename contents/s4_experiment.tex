% \section{実験}
% 本章では,提案手法を対象のゲームに適用し,
% \subsection*{実験設定}


% \subsection*{対象ゲーム: Star Collection}

\section{実験}\label{sec:experiment}
本章では,第\ref{chap:proposal}章で提案した分析および適用プロセスの有効性を検証するために実施した実験について述べる.本実験では,実際に BiFuzz を用いてゲームのテストを行い,その実行結果に対してロジスティック回帰,決定木,ランダムフォレストの3つの機械学習アルゴリズムを適用して予測モデルを構築する.構築した各モデルの性能を評価するために,データセットを用いた10分割交差検証(10-fold cross-validation)を行い,評価指標として適合率(Precision),再現率(Recall),F値(F1-score)を算出する.また,本実験における比較の基準(ベースライン)として,「すべてのテストケースで不具合を発見すると予測する(All Positive)」戦略を採用する.これは,予測モデルを用いずに網羅的にテストを実行する従来の運用(Run All)に相当する.本ベースラインの数値を提案手法が上回るかを検証することで,不具合発見の予測の正確性およびテスト効率化の有効性を評価する.
\subsection{実験設定}
本節では,実験におけるパラメータの設定および実行条件について述べる.
\subsubsection{対象パラメータの選定}
BiFuzz には,ゲーム全体の進行順序を決定する大域的ファジング(Global Fuzzing)と,移動経路の多様性を生み出す局所的ファジング(Local Fuzzing)の2種類のファジングが存在する.本研究の目的は,パラメータと不具合発見との関係性を明らかにすることであるが,今回の実験では特に\textbf{局所的ファジングのパラメータ}に焦点を当てて検証を行う.この理由は以下の通りである.第一に,本実験で検出対象として埋め込んだ不具合が「スタック(地形へのはまり)」であるためである.スタックは主に地形やオブジェクトとの微細な衝突判定によって発生するため,大まかなタスク順序を決定する大域的ファジングよりも,具体的な移動経路や経由地を決定する局所的ファジングの影響度が支配的であると考えられる.第二に,BiFuzz の仕様上,大域的ファジングにおけるチェックポイント(タスク位置)は固定されており,移動のバリエーションは主に局所的ファジングによって生成されるためである.
\subsubsection{パラメータ空間の限定}
実験においてパラメータが取りうる値は,連続値ではなく離散的な代表値に限定した.これは,全通りの組み合わせを実行することは時間的制約から現実的ではなく,かつパラメータの微小な値の変化(例:経由地数が1と2の違いなど)は不具合発見の結果に大きく影響しないと考えられるためである.具体的に採用したパラメータとその値の集合を以下に示す.
\begin{itemize}
    \item \textbf{cpNum}(経由地数): $\{1, 50, 99, 200, 300, 700\}$
    \item \textbf{cpNum_range}(経由地の生成範囲): $\{1, 50, 99\}$
    \item \textbf{cpNum_dir}(経由地の生成方向): $\{1, 2, 3, 4\}$
    \item \textbf{tree}(障害物となる木の数): $\{0, 500, 1000\}$
\end{itemize}

\subsubsection{実行回数}
BiFuzz はランダムな要素を含むため,同一のパラメータ設定であっても実行結果が異なる場合がある(非決定性).この影響を考慮し,結果の信頼性を担保するために,上記パラメータの組み合わせ1つにつき5回ずつ実行を行い,データを収集した.
\subsection{対象ゲーム}実験の対象となるビデオゲームとして,BiFuzz の提案論文\cite{Kato2024}でも使用された「STAR COLLECTION」を採用した.STAR COLLECTION は Unity ゲームエンジンを用いて開発された,簡易的なオープンワールドゲームである.ゲームの目的は,広大なマップ上に配置された「星のアイテム」をすべて回収することである.ゲーム内の環境はシンプルに構成されており,マップ上にはプレイヤキャラクタ,回収対象となる星のアイテム,そして移動の障害物となる木のオブジェクトのみが存在する.