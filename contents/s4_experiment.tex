\section{実験}\label{sec:experiment}
本章では,\chapref{sec:proposal}で提案した提案手法の有効性を検証するために実施した実験について述べる.

\figref{fig:testflow}に本実験の流れを示す.本実験では,BiFuzzに対して様々なパラメータについて5回ずつ適用してテストを実行し,不具合発見の成否を実験結果として記録する.
そして,収集したデータに対し,提案手法の有効性を評価するためのアルゴリズムとして,ロジスティック回帰分析,決定木,ランダムフォレストの3つを採用し,以下の3つの項目に関して予測モデルの構築を試みる.
\begin{itemize}
    \item 1回の実行ごとの分析
    \item 5回の実行のうち1回以上不具合を発見
    \item 5回の実行のうち5回とも不具合を発見
\end{itemize}
構築した各モデルの性能を評価するために,データセットを用いた10分割交差検証を行い,評価指標として適合率,再現率,F値を算出する.不具合発見の予測の正確性を比較するため,「すべてのテストケースで不具合を発見すると予測する」というベースラインを用意し,比較する.

\begin{figure}[H]
    \begin{center}
    \hspace{-15mm}
        \includegraphics[height=200mm]{img/test_method.png}
        \caption{実験の概要図}
        \label{fig:testflow}
    \end{center}
\end{figure}


\subsubsection*{実験目的}
本実験の目的は2つある.
1つ目は,不具合発見とマップ構造の相関を明らかにすることで,提案手法における適用プロセスの妥当性を検証することである.
2つ目は,パラメータ設定から不具合発見を予測するモデルの構築を試み,その予測精度および有効性を評価することである.また,これらの調査に向けて以下の2つのRQを設定した.

\begin{description}
    \item[RQ1:] \rqfirst
    \item[RQ2:] \rqsecond
\end{description}


\subsubsection*{実験設定}
% 本実験では局所ファジングのパラメータに焦点を当てて行う.これは本実験で検出対象として埋め込んだ不具合がスタックであるためである.スタックは主に地形やオブジェクトとの衝突判定によって発生するため,大まかなタスク順序を決定する大域的ファジングよりも,具体的な移動経路や経由地を決定する局所的ファジングの影響度が支配的であると考えられる.

本実験では,不具合発見の要因を分析するために,以下の計4つのパラメータを採用した.これらは,BiFuzzのテストケース生成に使用されるパラメータと,本実験のために独自に追加したマップに関するパラメータに大別される.

BiFuzzにおいて,テストケース(移動経路)の生成に直接関与する以下の3つのパラメータを採用した.
\begin{itemize}
    \item \textbf{cpNum}: 経由地の数
    \item \textbf{cpNum\_range}:経由地の生成範囲
    \item \textbf{cpNum\_dir}: 経由地の生成方向
\end{itemize}
これらを採用した理由は,BiFuzzの局所的ファジングにおいて経路決定に使用されるすべてのパラメータを網羅することで,パラメータが不具合発見に与える影響を漏れなく評価するためである.\\
マップ環境を変化させるためのパラメータとして,本実験独自に以下を採用した.
\begin{itemize}
    \item \textbf{tree}(障害物となる木の数): マップ上にランダムに配置される,衝突判定を持つ木のオブジェクトの総数.
\end{itemize}
これを採用した理由は,マップ構造(障害物の密度)の変化を特徴量として含めることで,RQ1で設定した「マップ構造と不具合発見の相関」を検証するためである.

実験においてパラメータが取りうる値は連続値ではなく離散的な代表値に限定した.これは,全通りの組み合わせを実行することは時間的制約から現実的ではなく,かつパラメータの微小な値の変化は不具合発見の結果に大きく影響しないと考えられるためである.具体的に採用したパラメータとその値の集合を以下に示す.
\begin{itemize}
    \item \textbf{cpNum} : $\{1, 50, 99, 200, 300, 700\}$
    \item \textbf{cpNum\_range} : $\{1, 50, 99\}$
    \item \textbf{cpNum\_dir} : $\{1, 2, 3, 4\}$
    \item \textbf{tree} : $\{0, 500, 1000\}$
\end{itemize}

また,BiFuzzは同一のパラメータ設定であっても実行結果が異なるような非決定性を含んでいる.この影響を考慮し,上記パラメータの組み合わせ1つにつき5回ずつ実行を行い,データを収集することとした.

今回の実験では\figref{fig:stack_bug_object}のようなスタックする不具合を模したオブジェクトをマップ上に100箇所ランダムに配置する.このオブジェクトは外側からかは侵入可能であるが,内側からは脱出不可能になっている.一定時間このオブジェクトの内側にいると「スタックした」と判定される.
\begin{figure}[H]
    \begin{center}
        \includegraphics{img/sample.png}
        \caption{スタックする不具合を模したオブジェクト}
        \label{fig:stack_bug_object}
    \end{center}
\end{figure}

\subsubsection*{対象ゲーム}実験の対象となるビデオゲームとして,BiFuzzの提案論文\cite{Kato2024}でも使用された「STAR COLLECTION」を採用した.本ゲームは ゲームエンジンであるUnity\footnote{\url{https://unity.com/ja}}を用いて開発された,簡易的なオープンワールドゲームである.ゲームの目的は,広大なマップ上に配置された星のアイテムをすべて回収することである.ゲーム内の環境はシンプルに構成されており,マップ上には\figref{fig:player}のようなプレイヤキャラクタ,\figref{fig:star}のような回収対象となる星のアイテム,そして\figref{fig:tree}のような移動の障害物となる木のオブジェクトのみが存在する.

\subsubsection*{分析の枠組み}前節で述べた通り,BiFuzzの実行結果には非決定性があり,同一のパラメータ設定であっても実行ごとに結果が異なる可能性がある.そこで本実験では,構築する予測モデルの堅牢性を多角的に評価するために,収集したデータに対して以下の3つの異なる粒度で正解ラベルを定義し,それぞれについて分析を行うこととした.

\begin{enumerate}
    \item \textbf{1回の実行ごとの分析 }:
    5回の試行をそれぞれ独立したデータとして扱い,「その1回の実行で不具合を発見できたか」を予測対象とする.
    
    \item \textbf{5回の実行のうち1回以上不具合を発見 }:
    パラメータの組み合わせごとに集計を行い,5回の試行のうち「少なくとも1回不具合を発見できた」場合を正とする.

    \item \textbf{5回の実行のうち5回とも不具合を発見 }:
    パラメータの組み合わせごとに集計を行い,5回の試行すべてにおいて不具合を発見できた場合のみを正とする.
\end{enumerate}

\begin{figure}[H]
    \begin{center}
        \includegraphics{img/sample.png}
        \caption{プレイヤキャラクタ}
        \label{fig:player}
    \end{center}
\end{figure}

\begin{figure}[H]
    \begin{center}
        \includegraphics{img/sample.png}
        \caption{星のオブジェクト}
        \label{fig:star}
    \end{center}
\end{figure}

\begin{figure}[H]
    \begin{center}
        \includegraphics{img/sample.png}
        \caption{木のオブジェクト}
        \label{fig:tree}
    \end{center}
\end{figure}

\subsection{1回の実行で不具合発見するか}
本節では,BiFuzzの1回の実行を1つのデータとし,その実行において不具合が発見されたか否かを予測対象とした場合の結果について述べる.

\subsubsection{ロジスティック回帰分析}
ロジスティック回帰分析を用いて導出された,不具合発見確率 $P$ を予測する回帰式を式(\ref{eq:logistic_single_split})に示す.

\begin{equation}
\begin{split}
    \label{eq:logistic_single_split}
    P &= \frac{1}{1 + \exp(-z)} \\
    \text{where, } z &= -1.788594 + 0.000146 \cdot tree + 0.001877 \cdot cpNum \\
      &\quad + 0.032893 \cdot cpNum\_range + 0.080988 \cdot cpNum\_dir
\end{split}
\end{equation}

ここで,各変数の係数に着目すると,$cpNum\_range$や $cpNum\_dir$の係数が正の値となっており,これらの値が増加するほど不具合発見確率が上昇する傾向が見られた.一方で,$tree$の係数は非常に小さく,不具合発見への寄与は限定的であると言える.

表\ref{tab:logistic_single_metrics}に,10分割交差検証によるモデルの評価結果を示す.加えて,比較対象として用意したベースラインの評価結果を表\ref{tab:baseline_result_1}に示す.表\ref{tab:baseline_result_1}より,ベースラインのF値は0.7313であるのに対し,ロジスティック回帰分析を用いた本モデルのF値の平均は0.8397となり,ベースラインを0.1以上上回る結果となった.ベースラインはすべてを陽性と予測するため再現率は1.0000となるが,適合率が0.5764と低い.一方,本モデルは適合率が0.7932とベースラインよりも約0.22高く,不具合発見の可能性が高いケースをより正確に識別できていることがわかる.
\begin{table}[H]
    \caption{ロジスティック回帰分析の評価結果(1回の実行)}
    \label{tab:logistic_single_metrics}
    \centering
    \begin{tabular}{lcccc}
        \hline
        評価指標 & 平均値 & 標準偏差 & 最小値 & 最大値 \\\hline 
        Accuracy (正解率) & 0.8036 & 0.0215 & 0.7704 & 0.8574 \\\hline
        Precision (適合率) & 0.7932 & 0.0199 & 0.7583 & 0.8431 \\\hline
        Recall (再現率) & 0.8922 & 0.0154 & 0.8732 & 0.9244 \\\hline
        F1 Score (F値) & 0.8397 & 0.0169 & 0.8158 & 0.8819 \\\hline
        AUC & 0.8288 & 0.0178 & 0.7994 & 0.8688 \\\hline
    \end{tabular}
\end{table}

\begin{table}[H]
  \centering
  \caption{ベースラインの評価結果 (1回の実行)}
  \label{tab:baseline_result_1}
  \begin{tabular}{l|r} \hline
    評価指標 & 値 \\ \hline \hline
    Accuracy (正解率) & 0.5764 \\
    Precision (適合率) & 0.5764 \\
    Recall (再現率) & 1.0000 \\
    F1 Score (F値) & 0.7313 \\ \hline
  \end{tabular}
\end{table}

\subsubsection{決定木}
決定木を用いてモデルを構築した際の特徴量重要度を表\ref{tab:dt_single_importance}に示す.
なお,本表に示す重要度は,評価指標の算出と同様に10分割交差検証を行い,各試行で得られた重要度の平均値を算出したものである.
結果より,$cpNum\_range$ の重要度が約69\%と最も高く,次いで $cpNum$ が約31\%を占めている.これら2つのパラメータで全体の重要度の99\%以上を占めているため,$tree$ と $cpNum\_dir$ の影響は極めて小さいと言える.
\begin{table}[H]
    \caption{決定木における特徴量重要度(1回の実行)}
    \label{tab:dt_single_importance}
    \centering
    \begin{tabular}{lrrr}\hline
        特徴量 & 平均重要度 & 標準偏差 & 割合(\%) \\\hline
        cpNum\_range & 0.6891 & 0.0035 & 68.91 \\\hline
        cpNum & 0.3083 & 0.0035 & 30.83 \\\hline
        tree & 0.0027 & 0.0004 & 0.27 \\\hline
        cpNum\_dir & 0.0000 & 0.0000 & 0.00 \\\hline
    \end{tabular}
\end{table}

表\ref{tab:dt_single_metrics}に,10分割交差検証による評価結果を示す.F値の平均は0.8631となり,ロジスティック回帰分析と比較して高い予測精度が得られた.
\begin{table}[H]
    \caption{決定木の評価結果(1回の実行)}
    \label{tab:dt_single_metrics}
    \centering
    \begin{tabular}{lcccc}\hline
    評価指標 & 平均値 & 標準偏差 & 最小値 & 最大値 \\\hline
    Accuracy (正解率) & 0.8451 & 0.0162 & 0.8167 & 0.8787 \\\hline
    Precision (適合率) & 0.8793 & 0.0158 & 0.8430 & 0.9085 \\\hline
    Recall (再現率) & 0.8477 & 0.0160 & 0.8266 & 0.8778 \\\hline
    F1 Score (F値) & 0.8631 & 0.0142 & 0.8403 & 0.8929 \\\hline
    AUC & 0.8648 & 0.0182 & 0.8303 & 0.9034\\\hline
    \end{tabular}
\end{table}

また,表\ref{tab:baseline_result_1}に示したベースラインのF値0.7313と比較すると,決定木モデルのF値の平均は0.8631であり,ベースラインを約0.13上回った.また,適合率に関してもベースラインの適合率0.5764と比較して,本モデルの適合率の平均は0.8793とベースラインよりも約0.30高く,不具合発見の可能性が高いケースをより正確に識別できていることがわかる.

\subsubsection{ランダムフォレスト}
ランダムフォレストを用いた場合の特徴量重要度を表\ref{tab:rf_single_importance}に示す.決定木と同様に,これらの値は10分割交差検証によって得られた各重要度の平均値を示している.
結果として,$cpNum\_range$ が最も高い重要度を示したが,その割合は約74\%と決定木の場合よりもさらに高くなっている.$cpNum$ は約26\%となっているため,決定木と同様にこれら2つの特徴量が支配的で$cpNum\_dir$と$tree$の影響は極めて小さい傾向があると言える.

\begin{table}[H]
    \caption{ランダムフォレストにおける特徴量重要度(1回の実行)}
    \label{tab:rf_single_importance}
    \centering
    \begin{tabular}{lrrr}\hline
        特徴量 & 平均重要度 & 標準偏差 & 割合(\%) \\\hline
        cpNum\_range & 0.7389 & 0.0066 & 73.89 \\\hline
        cpNum & 0.2582 & 0.0064 & 25.82 \\\hline
        tree & 0.0028 & 0.0004 & 0.28 \\\hline
        cpNum\_dir & 0.0000 & 0.0000 & 0.00 \\\hline
    \end{tabular}
\end{table}

表\ref{tab:rf_single_metrics}に,10分割交差検証による評価結果を示す.各評価指標の平均値は決定木の結果と同様の値を示しており,高い精度で不具合発見の予測が可能であることが確認された.
\begin{table}[H]
    \caption{ランダムフォレストの評価結果(1回の実行)}
    \label{tab:rf_single_metrics}
    \centering
    \begin{tabular}{lcccc}\hline
        評価指標 & 平均値 & 標準偏差 & 最小値 & 最大値 \\\hline
        Accuracy (正解率) & 0.8451 & 0.0162 & 0.8167 & 0.8787 \\\hline
        Precision (適合率) & 0.8793 & 0.0158 & 0.8430 & 0.9085 \\\hline
        Recall (再現率) & 0.8477 & 0.0160 & 0.8266 & 0.8778 \\\hline
        F1 Score (F値) & 0.8631 & 0.0142 & 0.8403 & 0.8929 \\\hline
        AUC & 0.8632 & 0.0171 & 0.8303 & 0.8976 \\\hline
    \end{tabular}
\end{table}

表\ref{tab:baseline_result_1}のベースラインとの比較においても,本モデルのF値の平均は0.8631であり,ベースラインのF値0.7313を大きく上回っている.ランダムフォレストは決定木と同様に0.8793と高い適合率となっており,ベースラインの適合率である0.5764を上回っている.

\subsection{\testsecond}
\subsubsection{ロジスティック回帰分析}

\subsubsection{決定木}
\subsubsection{ランダムフォレスト}

\subsection{\testthird}
\subsubsection{ロジスティック回帰分析}
\subsubsection{決定木}
\subsubsection{ランダムフォレスト}

\subsection{実験の考察}