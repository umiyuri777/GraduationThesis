\section{実験}\label{sec:experiment}
本章では,\chapref{sec:proposal}で提案した提案手法の有効性を検証するために実施した実験について述べる.

\figref{fig:testflow}に本実験の流れを示す.本実験では,BiFuzzに対して様々なパラメータについて5回ずつ適用してテストを実行し,不具合発見の成否を実験結果として記録する.
そして,収集したデータに対し,提案手法の有効性を評価するためのアルゴリズムとして,ロジスティック回帰分析,決定木,ランダムフォレストの3つを採用し,以下の3つの項目に関して予測モデルの構築を試みる.
\begin{itemize}
    \item \testfirst
    \item \testsecond
    \item \testthird
\end{itemize}
構築した各モデルの性能を評価するために,データセットを用いた10分割交差検証を行い,評価指標として適合率,再現率,F値を算出する.不具合発見の予測の正確性を比較するため,「すべてのテストケースで不具合を発見すると予測する」というベースラインを用意し,比較する.

\begin{figure}[H]
    \begin{center}
        \includegraphics[height=200mm]{img/test_method_1.png}
        \caption{実験の概要図}
        \label{fig:testflow}
    \end{center}
\end{figure}

\subsubsection*{実験目的}
本実験の目的は2つある.
1つ目は,パラメータ設定から不具合発見を予測するモデルの構築を試み,その予測精度および有効性を評価することである.
2つ目は,不具合発見とマップ構造の相関を明らかにすることで,提案手法における適用プロセスの妥当性を検証することである.また,これらの調査に向けて以下の2つのRQを設定した.

\begin{description}
    \item[RQ1:] \rqfirst
    \item[RQ2:] \rqsecond
\end{description}


\subsubsection*{実験設定}
% 本実験では局所ファジングのパラメータに焦点を当てて行う.これは本実験で検出対象として埋め込んだ不具合がスタックであるためである.スタックは主に地形やオブジェクトとの衝突判定によって発生するため,大まかなタスク順序を決定する大域的ファジングよりも,具体的な移動経路や経由地を決定する局所的ファジングの影響度が支配的であると考えられる.

本実験では,不具合発見の要因を分析するために,以下の計4つのパラメータを採用した.これらは,BiFuzzのテストケース生成に使用されるパラメータと,本実験のために独自に追加したマップに関するパラメータに大別される.

BiFuzzにおいて,テストケース(移動経路)の生成に直接関与する以下の3つのパラメータをすべて採用した.
\begin{itemize}
    \item \textbf{cpNum}: 経由地の数
    \item \textbf{cpNum\_range}:経由地の生成範囲
    \item \textbf{cpNum\_dir}: 経由地の生成方向
\end{itemize}
これらを採用した理由は,BiFuzzの局所的ファジングにおいて経路決定に使用されるすべてのパラメータを網羅することで,パラメータが不具合発見に与える影響を漏れなく評価するためである.\\
マップ環境を変化させるためのパラメータとして,本実験で独自に以下を採用した.
\begin{itemize}
    \item \textbf{tree}(障害物となる木の数): マップ上にランダムに配置される,衝突判定を持つ木のオブジェクトの総数.
\end{itemize}
これを採用した理由は,マップ構造(障害物の密度)の変化を特徴量として含めることで,マップ構造と不具合発見の相関を検証するためである.

実験においてパラメータが取りうる値は連続値ではなく離散的な代表値に限定した.これは,全通りの組み合わせを実行することは時間的制約から現実的ではなく,かつパラメータの微小な値の変化は不具合発見の結果に大きく影響しないと考えられるためである.具体的に採用したパラメータとその値の集合を以下に示す.
\begin{itemize}
    \item \textbf{cpNum} : $\{1, 50, 99, 200, 300, 700\}$
    \item \textbf{cpNum\_range} : $\{1, 50, 99\}$
    \item \textbf{cpNum\_dir} : $\{1, 2, 3, 4\}$
    \item \textbf{tree} : $\{0, 500, 1000\}$
\end{itemize}

また,BiFuzzは同一のパラメータ設定であっても実行結果が異なるような非決定性を含んでいる.この影響を考慮し,上記パラメータの組み合わせ1つにつき5回ずつ実行を行い,データを収集することとした.

今回の実験では\figref{fig:stack_bug_object}のようなスタックする不具合を模したオブジェクトをマップ上に100箇所ランダムに配置する.このオブジェクトは外側からかは侵入可能であるが,内側からは脱出不可能になっている.一定時間このオブジェクトの内側にいると「スタックした」と判定される.
\begin{figure}[H]
    \begin{center}
        \includegraphics{img/sample.png}
        \caption{スタックする不具合を模したオブジェクト}
        \label{fig:stack_bug_object}
    \end{center}
\end{figure}

\subsubsection*{対象ゲーム}実験の対象となるビデオゲームとして,BiFuzzの提案論文\cite{Kato2024}でも使用された「STAR COLLECTION」を採用した.本ゲームは ゲームエンジンであるUnity\footnote{\url{https://unity.com/ja}}を用いて開発された,簡易的なオープンワールドゲームである.ゲームの目的は,広大なマップ上に配置された星のアイテムをすべて回収することである.ゲーム内の環境はシンプルに構成されており,マップ上には\figref{fig:player}のようなプレイヤキャラクタ,\figref{fig:star}のような回収対象となる星のアイテム,そして\figref{fig:tree}のような移動の障害物となる木のオブジェクトのみが存在する.

\subsubsection*{分析の枠組み}前節で述べた通り,BiFuzzの実行結果には非決定性があり,同一のパラメータ設定であっても実行ごとに結果が異なる可能性がある.そこで本実験では,構築する予測モデルの堅牢性を多角的に評価するために,収集したデータに対して以下の3つの異なる粒度で正例を定義し,それぞれについて分析を行うこととした.

\begin{enumerate}
    \item \textbf{1回の実行ごとの分析 }:
    5回の試行をそれぞれ独立したデータとして扱い,1回の実行で不具合を発見できたかを予測対象とする.これにより,非決定性を含みながらもモデルが不具合発見するパラメータを特定できるか調査する.
    
    \item \textbf{5回の実行のうち1回以上不具合を発見 }:
    パラメータの組み合わせごとに集計を行い,5回の試行のうち少なくとも1回不具合を発見できた場合を正例とする.これにより,少しでも不具合発見する可能性のあるパラメータ網羅的に特定できるかを調査する.

    \item \textbf{5回の実行のうち5回とも不具合を発見 }:
    パラメータの組み合わせごとに集計を行い,5回の試行すべてにおいて不具合を発見できた場合のみを正例とする.これにより,高い再現性で不具合発見できるパラメータを特定できるかを調査する.
\end{enumerate}

\begin{figure}[H]
    \begin{center}
        \includegraphics{img/player.png}
        \caption{プレイヤキャラクタ}
        \label{fig:player}
    \end{center}
\end{figure}

\begin{figure}[H]
    \begin{center}
        \includegraphics{img/star.png}
        \caption{星のオブジェクト}
        \label{fig:star}
    \end{center}
\end{figure}

\begin{figure}[H]
    \begin{center}
        \includegraphics{img/tree.png}
        \caption{木のオブジェクト}
        \label{fig:tree}
    \end{center}
\end{figure}

\subsection{1回の実行で不具合発見するか}\label{chap:test1}
本節では,BiFuzzの1回の実行を1つのデータとし,その実行において不具合が発見されたか否かを予測対象とした場合の結果について述べる.
まず,比較対象として用意した本設定におけるベースラインの評価結果を表\ref{tab:baseline_result_1}に,今回のモデル構築に使用したデータセットの分布を表\ref{data_set_1}に示す.

\begin{table}[H]
  \centering
  \caption{ベースラインの評価結果 (\testfirst)}
  \label{tab:baseline_result_1}
  \begin{tabular}{l|r} \hline
    評価指標 & 値 \\ \hline \hline
    正解率 & 0.5764 \\
    適合率 & 0.5764 \\
    再現率 & 1.0000 \\
    F値 & 0.7313 \\ \hline
  \end{tabular}
\end{table}

\begin{table}[H]
  \centering
  \caption{データセットの分布(\testfirst)}
  \label{data_set_1}
  \begin{tabular}{l|r} \hline
    分類 & 数 \\ \hline \hline
    不具合発見する & 6223\\
    不具合発見しない & 4577\\ \hline
  \end{tabular}
\end{table}

\paragraph{予測モデルの精度比較}
ロジスティック回帰分析,決定木,ランダムフォレストの3つのモデルの10分割交差検証による評価結果を表\ref{tab:logistic_single_metrics},表\ref{tab:dt_single_metrics},表\ref{tab:rf_single_metrics}に示す.

まず,ベースラインとの比較を行う.表\ref{tab:baseline_result_1}に示したベースラインのF値は0.7313であるのに対し,表\ref{tab:logistic_single_metrics}に示すロジスティック回帰分析のF値平均は0.8397,表\ref{tab:dt_single_metrics}に示す決定木および表\ref{tab:rf_single_metrics}に示すランダムフォレストのF値平均は0.8631となり,いずれのモデルもベースラインを上回る結果となった.
特に適合率に着目すると,ベースラインが0.5764であるのに対し,ロジスティック回帰分析では0.7932,決定木とランダムフォレストでは約0.8793となった.これは,提案手法が不具合発見の可能性が高いケースを正確に識別できていることを示している.

次に,モデル間での比較を行う.決定木とランダムフォレストのF値平均は0.8631であり,ロジスティック回帰分析のF値0.8397と比較して約0.02高く,線型モデルよりも非線形モデルの方が,より高い精度で不具合発見を予測できていることが確認された.

\begin{table}[H]
    \caption{ロジスティック回帰分析の評価結果(\testfirst)}
    \label{tab:logistic_single_metrics}
    \centering
    \begin{tabular}{lcccc}
        \hline
        評価指標 & 平均値 & 標準偏差 & 最小値 & 最大値 \\\hline \hline
        正解率& 0.8036 & 0.0215 & 0.7704 & 0.8574 \\\hline
        適合率 & 0.7932 & 0.0199 & 0.7583 & 0.8431 \\\hline
        再現率 & 0.8922 & 0.0154 & 0.8732 & 0.9244 \\\hline
        F値 & 0.8397 & 0.0169 & 0.8158 & 0.8819 \\\hline
        AUC & 0.8288 & 0.0178 & 0.7994 & 0.8688 \\\hline
    \end{tabular}
\end{table}

\begin{table}[H]
    \caption{決定木の評価結果(\testfirst)}
    \label{tab:dt_single_metrics}
    \centering
    \begin{tabular}{lcccc}\hline
    評価指標 & 平均値 & 標準偏差 & 最小値 & 最大値 \\\hline \hline
    正解率 & 0.8451 & 0.0162 & 0.8167 & 0.8787 \\\hline
    適合率 & 0.8793 & 0.0158 & 0.8430 & 0.9085 \\\hline
    再現率 & 0.8477 & 0.0160 & 0.8266 & 0.8778 \\\hline
    F値 & 0.8631 & 0.0142 & 0.8403 & 0.8929 \\\hline
    AUC & 0.8648 & 0.0182 & 0.8303 & 0.9034\\\hline
    \end{tabular}
\end{table}

\begin{table}[H]
    \caption{ランダムフォレストの評価結果(\testfirst)}
    \label{tab:rf_single_metrics}
    \centering
    \begin{tabular}{lcccc}\hline 
        評価指標 & 平均値 & 標準偏差 & 最小値 & 最大値 \\\hline \hline
        正解率 & 0.8451 & 0.0162 & 0.8167 & 0.8787 \\\hline
        適合率 & 0.8793 & 0.0158 & 0.8430 & 0.9085 \\\hline
        再現率 & 0.8477 & 0.0160 & 0.8266 & 0.8778 \\\hline
        F値 & 0.8631 & 0.0142 & 0.8403 & 0.8929 \\\hline
        AUC & 0.8632 & 0.0171 & 0.8303 & 0.8976 \\\hline
    \end{tabular}
\end{table}

\paragraph{特徴量の分析}
各モデルにおいて,どのパラメータが不具合発見に寄与したかを分析する.

まず,ロジスティック回帰分析について述べる.各変数のスケールの違いによる影響を排除し,各変数の重要度を直接比較するために,全変数を標準化した上で回帰分析を行った.
標準化された変数を用いた回帰式を式(\ref{eq:logistic_single_scaled})に示す.

\begin{equation}
\begin{split}
    \label{eq:logistic_single_scaled}
    P &= \frac{1}{1 + \exp(-z)} \\
    \text{where, } z &= 0.445 + 0.060 \cdot tree_{std} + 0.439 \cdot cpNum_{std} \\
      &\quad + 1.316 \cdot cpNum\_range_{std} + 0.046 \cdot cpNum\_dir_{std}
\end{split}
\end{equation}

ここで,添字 $std$ は各変数が平均 0,分散 1 に標準化されていることを示す.各変数の標準化は式(\ref{eq:standardization_1})によって行う.

\begin{equation}
    \label{eq:standardization_1}
    x_{std} = \frac{x - \mu}{\sigma} \quad
    \left(
    \begin{aligned}
        &tree: \mu=500.0, \sigma \approx 408.2 \\
        &cpNum: \mu=225.0, \sigma \approx 234.0 \\
        &cpNum\_range: \mu=50.0, \sigma \approx 40.0 \\
        &cpNum\_dir: \mu=1.15, \sigma \approx 0.57
    \end{aligned}
    \right)
\end{equation}

式(\ref{eq:logistic_single_scaled})の標準化偏回帰係数に着目すると,$cpNum\_range_{std}$ の係数が $1.316$ と最も大きな値をとり,次いで $cpNum_{std}$ が $0.439$ となっている.これは,不具合発見率の上昇には $cpNum\_range$が最も寄与しており,次いで $cpNum$ が寄与している.一方で,$tree_{std}$ の標準化偏回帰係数は $0.060$, $cpNum\_dir_{std}$ は $0.046$ となっており,この2つの係数は相対的に小さく,これらのパラメータが不具合発見率の上昇に与える影響は限定的であると言える.

次に,決定木およびランダムフォレストの十分割交差検証による特徴量重要度を表\ref{tab:dt_single_importance}および表\ref{tab:rf_single_importance}に示す.
両モデルとも共通して $cpNum\_range$ が約69\%〜74\%と最も高い重要度を示し,次いで $cpNum$ が約26\%〜31\%という重要度を示した.これら2つのパラメータで重要度全体の99\%以上を占めており,$tree$ や $cpNum\_dir$ の影響は無視できるほど小さい.

以上の結果から,使用するアルゴリズムに関わらず,$cpNum\_range$ と  $cpNum$の経路生成に関するパラメータが不具合発見に大きく影響していることが示された.また,$tree$ が不具合発見に与える影響が少なかったことから,マップ内の障害物数による影響は極めて小さいという共通した傾向が明らかになった.

\begin{table}[H]
    \caption{決定木における特徴量重要度(\testfirst)}
    \label{tab:dt_single_importance}
    \centering
    \begin{tabular}{lrrrr}\hline
        特徴量 & 平均重要度 & 標準偏差 & 最小値 & 最大値 \\\hline \hline
        cpNum\_range & 0.6891 & 0.0035 & 0.6802 & 0.6925\\\hline
        cpNum & 0.3083 & 0.0035 & 0.3042 & 0.3166 \\\hline
        tree & 0.0027 & 0.0004 & 0.0019 & 0.0034 \\\hline
        cpNum\_dir & 0.0000 & 0.0000 & 0.0000 & 0.0000 \\\hline
    \end{tabular}
\end{table}

\begin{table}[H]
    \caption{ランダムフォレストにおける特徴量重要度(\testfirst)}
    \label{tab:rf_single_importance}
    \centering
    \begin{tabular}{lrrrr}\hline 
        特徴量 & 平均重要度 & 標準偏差 & 最小値 & 最大値 \\\hline \hline
        cpNum\_range & 0.7389 & 0.0066 & 0.7277 & 0.7510\\\hline
        cpNum & 0.2582 & 0.0064 & 0.2467 & 0.2694 \\\hline
        tree & 0.0028 & 0.0004 & 0.0023 & 0.0038 \\\hline
        cpNum\_dir & 0.0000 & 0.0000 & 0.0000 & 0.0000 \\\hline
    \end{tabular}
\end{table}


\subsection{5回の実行のうち1回以上不具合発見するか}\label{chap:test2}
本節では,BiFuzzのパラメータ設定ごとに5回実行を行った際,少なくとも1回以上不具合が発見されたか否かを予測対象とした場合の結果について述べる.
まず,比較対象として用意した本設定におけるベースラインの評価結果を表\ref{tab:baseline_result_5}に,今回のモデル構築に使用したデータセットの分布を表\ref{data_set_2}に示す.

\begin{table}[H]
  \centering
  \caption{ベースラインの評価結果 (\testsecond)}
  \label{tab:baseline_result_5}
  \begin{tabular}{l|r} \hline
    評価指標 & 値 \\ \hline \hline
    正解率 & 0.8449 \\
    適合率 & 0.8449 \\
    再現率 & 1.0000 \\
    F値 & 0.9159 \\ \hline
  \end{tabular}
\end{table}

\begin{table}[H]
  \centering
  \caption{データセットの分布(\testsecond)}
  \label{data_set_2}
  \begin{tabular}{l|r} \hline
    分類 & 数 \\ \hline \hline
    5回実行のうち1回でも不具合発見する & 1824\\
    不具合発見しない & 336\\ \hline
  \end{tabular}
\end{table}

\paragraph{予測モデルの精度比較}
ロジスティック回帰分析,決定木,ランダムフォレストの3つのモデルの10分割交差検証による評価結果を表\ref{tab:logistic_multi_metrics},表\ref{tab:dt_multi_metrics},表\ref{tab:rf_multi_metrics}に示す.
表\ref{tab:baseline_result_5}に示したベースラインのF値は0.9159であるのに対し,ロジスティック回帰分析および決定木のF値平均は0.9157,ランダムフォレストのF値平均は0.9145となり,いずれの手法を用いてもベースラインと比較して予測性能の向上は見られなかった.

また,再現率に着目すると,ロジスティック回帰分析と決定木では1.0000,ランダムフォレストでも0.9973と極めて高い値を示している.これは,すべてのテストケースに対して「不具合を発見する」と予測するベースラインとほぼ同等の挙動であることを意味する.
この要因として,表\ref{data_set_2}に示したデータセットの不均衡が挙げられる.「5回のうち1回以上不具合を発見する」という条件を満たす正例は全体の約84\%を占めており,学習されたモデルは特徴量に関わらず不具合発見と予測する傾向が強まったため,ベースラインとの差異が生じなかったと考えられる.

\begin{table}[H]
    \caption{ロジスティック回帰分析の評価結果(\testsecond)}
    \label{tab:logistic_multi_metrics}
    \centering
    \begin{tabular}{lcccc}
        \hline
        評価指標 & 平均値 & 標準偏差 & 最小値 & 最大値 \\\hline \hline
        正解率 & 0.8444 & 0.0023 & 0.8426 & 0.8472 \\\hline
        適合率 & 0.8444 & 0.0023 & 0.8426 & 0.8472 \\\hline
        再現率 & 1.0000 & 0.0000 & 1.0000 & 1.0000 \\\hline
        F値 & 0.9157 & 0.0013 & 0.9146 & 0.9173 \\\hline
        AUC & 0.8228 & 0.0227 & 0.7871 & 0.8653 \\\hline
    \end{tabular}
\end{table}

\begin{table}[H]
    \caption{決定木の評価結果(\testsecond)}
    \label{tab:dt_multi_metrics}
    \centering
    \begin{tabular}{lcccc}\hline
    評価指標 & 平均値 & 標準偏差 & 最小値 & 最大値 \\\hline \hline
    正解率 & 0.8444 & 0.0023 & 0.8426 & 0.8472 \\\hline
    適合率 & 0.8444 & 0.0023 & 0.8426 & 0.8472 \\\hline
    再現率 & 1.0000 & 0.0000 & 1.0000 & 1.0000 \\\hline
    F値 & 0.9157 & 0.0013 & 0.9146 & 0.9173 \\\hline
    AUC& 0.8381 & 0.0193 & 0.8084 & 0.8683 \\\hline
    \end{tabular}
\end{table}

\begin{table}[H]
    \caption{ランダムフォレストの評価結果(\testsecond)}
    \label{tab:rf_multi_metrics}
    \centering
    \begin{tabular}{lcccc}\hline
        評価指標 & 平均値 & 標準偏差 & 最小値 & 最大値 \\\hline \hline
        正解率 & 0.8426 & 0.0041 & 0.8333 & 0.8472 \\\hline
        適合率 & 0.8445 & 0.0021 & 0.8426 & 0.8472 \\\hline
        再現率 & 0.9973 & 0.0056 & 0.9835 & 1.0000 \\\hline
        F値 & 0.9145 & 0.0025 & 0.9086 & 0.9173 \\\hline
        AUC & 0.8385 & 0.0237 & 0.7933 & 0.8767 \\\hline
    \end{tabular}
\end{table}


\paragraph{特徴量の分析}
次に,予測における各パラメータの影響度について分析する.
ロジスティック回帰分析については,前節と同様に全変数を標準化した上で回帰分析を行った.標準化された変数を用いた回帰式を式(\ref{eq:logistic_multi_scaled})に示す.

\begin{equation}
\begin{split}
    \label{eq:logistic_multi_scaled}
    P &= \frac{1}{1 + \exp(-z)} \\
    \text{where, } z &= 0.684456 + 0.096883 \cdot tree_{std} + 0.532612 \cdot cpNum_{std} \\
      &\quad + 1.420297 \cdot cpNum\_range_{std} + 0.108988 \cdot cpNum\_dir_{std}
\end{split}
\end{equation}

ここで,添字 $std$ は各変数が平均 0,分散 1 に標準化されていることを示す.各変数の標準化は式(\ref{eq:standardization_1})によって行う.

式(\ref{eq:logistic_multi_scaled})の標準化偏回帰係数を確認すると,$cpNum\_range_{std}$ の係数が $1.420297$ と最も大きく,次いで $cpNum_{std}$ が $0.532612$ となっている.\cref{chap:test1}の結果と同様に,これらの値が増加するほど不具合発見確率が上昇する傾向が見られた.一方で,$tree_{std}$ と $cpNum\_dir_{std}$ の係数はそれぞれ $0.096883$, $0.108988$ と相対的に小さく,不具合発見への寄与は限定的である.

決定木およびランダムフォレストにおける特徴量重要度を表\ref{tab:dt_multi_importance}および表\ref{tab:rf_multi_importance}に示す.
決定木では $cpNum\_range$ が約77\%,ランダムフォレストでは約72\%と高い重要度を示しており,次いで $cpNum$ が20\%〜24\%程度を占めている.
以上の結果より,\cref{chap:test1}と同様に,$cpNum\_range$と$cpNum$が不具合発見に対して支配的な要因であることが確認された.

\begin{table}[H]
    \caption{決定木における特徴量重要度(\testsecond)}
    \label{tab:dt_multi_importance}
    \centering
    \begin{tabular}{lrrrr}\hline
        特徴量 & 平均重要度 & 標準偏差 & 最小値 & 最大値 \\\hline \hline
        cpNum\_range & 0.7692 & 0.0119 & 0.7467 & 0.7886 \\\hline
        cpNum & 0.2238 & 0.0159 & 0.1910 & 0.2486 \\\hline
        tree & 0.0070 & 0.0073 & 0.0036 & 0.0275 \\\hline
        cpNum\_dir & 0.0000 & 0.0000 & 0.0000 & 0.0000 \\\hline
    \end{tabular}
\end{table}

\begin{table}[H]
    \caption{ランダムフォレストにおける特徴量重要度(\testsecond)}
    \label{tab:rf_multi_importance}
    \centering
    \begin{tabular}{lrrrr}\hline
        特徴量 & 平均重要度 & 標準偏差 & 最小値 & 最大値 \\\hline \hline
        cpNum\_range & 0.7163 & 0.0141 & 0.6979 & 0.7379 \\\hline
        cpNum & 0.2378 & 0.0142 & 0.2073 & 0.2556 \\\hline
        tree & 0.0348 & 0.0036 & 0.0298 & 0.0426 \\\hline
        cpNum\_dir & 0.0111 & 0.0023 & 0.0073 & 0.0150 \\\hline
    \end{tabular}
\end{table}





\subsection{5回の実行のうち5回とも不具合発見するか}\label{chap:test3}
本節では,BiFuzzのパラメータ設定ごとに5回実行を行った際,5回の試行すべてにおいて不具合が発見されたか否かを予測対象とした場合の結果について述べる.
まず,比較対象として用意した本設定におけるベースラインの評価結果を表\ref{tab:baseline_result_all}に,今回のモデル構築に使用したデータセットの分布を表\ref{data_set_3}に示す.

\begin{table}[H]
  \centering
  \caption{ベースラインの評価結果(\testthird)}
  \label{tab:baseline_result_all}
  \begin{tabular}{l|r} \hline
    評価指標 & 値 \\ \hline \hline
    正解率 & 0.3542 \\
    適合率 & 0.3542 \\
    再現率 & 1.0000 \\
    F値 & 0.5231 \\ \hline
  \end{tabular}
\end{table}

\begin{table}[H]
  \centering
  \caption{データセットの分布(\testthird)}
  \label{data_set_3}
  \begin{tabular}{l|r} \hline
    分類 & 数 \\ \hline \hline
    5回実行のうち5回全て不具合発見する & 763\\
    不具合発見しない & 1397\\ \hline
  \end{tabular}
\end{table}

\paragraph{予測モデルの精度比較}
ロジスティック回帰分析,決定木,ランダムフォレストの3つのモデルの10分割交差検証による評価結果を表\ref{tab:logistic_all_metrics},表\ref{tab:dt_all_metrics},表\ref{tab:rf_all_metrics}に示す.

本実験設定において,表\ref{tab:baseline_result_all}に示すベースラインのF値は0.5231である.これに対し,ロジスティック回帰分析のF値平均は0.6648となり,ベースラインを約0.14上回る結果となった.
さらに,決定木およびランダムフォレストのF値平均は約0.77であり,ベースラインと比較して約0.25の精度向上が確認された.

特に決定木の再現率に着目すると0.9882という高い値を示している.これは,5回とも不具合を発見できる再現性の高いパラメータ設定を,モデルが高い精度で予測できていることを意味する.

\begin{table}[H]
    \caption{ロジスティック回帰分析の評価結果(\testthird)}
    \label{tab:logistic_all_metrics}
    \centering
    \begin{tabular}{lcccc}
        \hline
        評価指標 & 平均値 & 標準偏差 & 最小値 & 最大値 \\\hline \hline
        正解率 & 0.7306 & 0.0308 & 0.6713 & 0.7778 \\\hline
        適合率 & 0.5949 & 0.0390 & 0.5238 & 0.6591 \\\hline
        再現率 & 0.7561 & 0.0527 & 0.6974 & 0.8701 \\\hline
        F値 & 0.6648 & 0.0346 & 0.6077 & 0.7101 \\\hline
        AUC & 0.6648 & 0.0346 & 0.6077 & 0.7101 \\\hline
    \end{tabular}
\end{table}

\begin{table}[H]
    \caption{決定木の評価結果(\testthird)}
    \label{tab:dt_all_metrics}
    \centering
    \begin{tabular}{lcccc}\hline
    評価指標 & 平均値 & 標準偏差 & 最小値 & 最大値 \\\hline \hline
    正解率 & 0.7981 & 0.0335 & 0.7500 & 0.8565 \\\hline
    適合率 & 0.6412 & 0.0402 & 0.5846 & 0.7143 \\\hline
    再現率 & 0.9882 & 0.0123 & 0.9610 & 1.0000 \\\hline
    F値 & 0.7769 & 0.0284 & 0.7379 & 0.8287 \\\hline
    AUC & 0.8682 & 0.0215 & 0.8504 & 0.9210 \\\hline
    \end{tabular}
\end{table}

\begin{table}[H]
    \caption{ランダムフォレストの評価結果(\testthird)}
    \label{tab:rf_all_metrics}
    \centering
    \begin{tabular}{lcccc}\hline
        評価指標 & 平均値 & 標準偏差 & 最小値 & 最大値 \\\hline \hline
        正解率 & 0.7995 & 0.0299 & 0.7546 & 0.8519 \\\hline
        適合率 & 0.6466 & 0.0373 & 0.5891 & 0.7115 \\\hline
        再現率 & 0.9659 & 0.0257 & 0.9342 & 1.0000 \\\hline
        F値 & 0.7738 & 0.0262 & 0.7396 & 0.8222 \\\hline
        AUC & 0.8385 & 0.0237 & 0.7933 & 0.8767 \\\hline
    \end{tabular}
\end{table}

\paragraph{特徴量の分析}
各パラメータの影響度について分析する.
ロジスティック回帰分析については,全変数を標準化した上で回帰分析を行った.標準化された変数を用いた回帰式を式(\ref{eq:logistic_all_scaled})に示す.

\begin{equation}
\begin{split}
    \label{eq:logistic_all_scaled}
    P &= \frac{1}{1 + \exp(-z)} \\
    \text{where, } z &= -0.283659 + 0.128433 \cdot tree_{std} + 0.470744 \cdot cpNum_{std} \\
      &\quad + 1.433006 \cdot cpNum\_range_{std} - 0.010976 \cdot cpNum\_dir_{std}
\end{split}
\end{equation}

式(\ref{eq:logistic_all_scaled})の標準化偏回帰係数を確認すると, $cpNum\_range_{std}$ の係数が $1.433006$ と最も大きな正の値を示しており,次いで $cpNum_{std}$ が 0.470744 となっている.一方で,$tree_{std}$ は 0.128433 と小さく,生成方向を示す $cpNum\_dir_{std}$ は -0.010976 と0に近い負の値となっている.

決定木およびランダムフォレストにおける特徴量重要度を表\ref{tab:dt_all_importance}および表\ref{tab:rf_all_importance}に示す.
決定木では $cpNum\_range$ が約65\%,ランダムフォレストでは約63\%の重要度を占め,次いで $cpNum$ が約33\%を占めている.
これまでの分析結果と同様に,再現性高く不具合を発見するためには,経路生成範囲である $cpNum\_range$ と経由地数である $cpNum$ が主要な要因であり,マップ構造である $tree$ や生成方向である $cpNum\_dir$ の影響は小さいことが確認された.

\begin{table}[H]
    \caption{決定木における特徴量重要度(\testthird)}
    \label{tab:dt_all_importance}
    \centering
    \begin{tabular}{lrrrr}\hline
        特徴量 & 平均重要度 & 標準偏差 & 最小値 & 最大値 \\\hline \hline
        cpNum\_range & 0.6536 & 0.0077 & 0.6419 & 0.6631 \\\hline
        cpNum & 0.3321 & 0.0069 & 0.3235 & 0.3417 \\\hline
        tree & 0.0141 & 0.0017 & 0.0114 & 0.0166 \\\hline
        cpNum\_dir & 0.0003 & 0.0004 & 0.0000 & 0.0012 \\\hline
    \end{tabular}
\end{table}

\begin{table}[H]
    \caption{ランダムフォレストにおける特徴量重要度(\testthird)}
    \label{tab:rf_all_importance}
    \centering
    \begin{tabular}{lrrrr}\hline
        特徴量 & 平均重要度 & 標準偏差 & 最小値 & 最大値 \\\hline \hline
        cpNum\_range & 0.6262 & 0.0064 & 0.6147 & 0.6371 \\\hline
        cpNum & 0.3153 & 0.0054 & 0.3069 & 0.3242 \\\hline
        cpNum\_dir & 0.0357 & 0.0012 & 0.0336 & 0.0371 \\\hline
        tree & 0.0229 & 0.0013 & 0.0210 & 0.0250 \\\hline
    \end{tabular}
\end{table}


\subsection{考察}\label{sec:discussion}
本節では,実験結果に基づき,実験目的において設定した2つのRQに対する回答を述べる.

\subsubsection*{RQ1: \rqfirst}
本実験では,異なる3つの粒度で予測モデルの構築を試みた.
\cref{chap:test1}における1回の実行を対象とした分析,および\cref{chap:test3}における5回の実行すべてで不具合を発見する場合を対象とした分析において,提案手法はベースラインを上回るF値を達成した.特に,再現性の高いパラメータ設定を予測する\cref{chap:test3}の設定では,決定木やランダムフォレストを用いることで,ベースラインと比較して大幅な精度向上が確認された.これらの結果は,提案手法によって不具合発見の成否を高精度に予測するモデルが構築可能であることを示している.

一方で,\cref{chap:test2}における5回の実行のうち1回以上不具合を発見する場合を対象とした分析では,提案手法はベースラインを上回る結果を示さなかった.この要因は,収集した実験データの分布にある.当該設定では,不具合を発見できたテストケースが全体の約84\%を占めていた.このように正例の割合が高い不均衡なデータセットであったため,すべてを正例と予測するベースラインの適合率が高くなり,予測モデルとの差異が顕著に現れなかったと考えられる.
以上のことから,データの偏りが少ないという条件下においては,提案手法によって有効な予測モデルが構築可能であると結論付ける.

\subsubsection*{RQ2: \rqsecond}
すべての実験設定および分析手法において,不具合発見率に支配的な影響を与えているのは,経路生成範囲を示す $cpNum\_range$ および経由地数を示す $cpNum$ であった.一方で,障害物の数を示す $tree$ および生成方向を示す $cpNum\_dir$ の寄与度は相対的に低く,不具合発見に対する影響は限定的であった.

この結果は,本実験で対象としたスタックという不具合に関して,マップ上の障害物密度の変化は発見率に大きな影響を与えないことを示唆している.
提案手法における適用プロセスは,ゲームになんらかの変更が加わった際に実施されることを想定している.この変更にはマップ構造の変化も含まれるが,本実験の結果より,マップ構造の変化は不具合発見に大きな影響を与えないことが確認された.したがって,マップ構造が変化した場合であっても,経路生成に関するパラメータを重点的に探索・調整するという提案手法の方針は妥当であり,その適用プロセスの有効性が示されたといえる.