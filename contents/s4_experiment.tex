\section{実験}\label{sec:experiment}
本章では,\chapref{sec:proposal}で提案した提案手法の有効性を検証するために実施した実験について述べる.

\figref{fig:testflow}に本実験の流れを示す.本実験では,BiFuzzに対して様々なパラメータについて5回ずつ適用してテストを実行し,不具合発見の成否を実験結果として記録する.
そして,収集したデータに対し,提案手法の有効性を評価するためのアルゴリズムとして,ロジスティック回帰分析,決定木,ランダムフォレストの3つを採用し,以下の3つの項目に関して予測モデルの構築を試みる.
\begin{itemize}
    \item 1回の実行ごとの分析
    \item 5回の実行のうち1回以上不具合を発見
    \item 5回の実行のうち5回とも不具合を発見
\end{itemize}
構築した各モデルの性能を評価するために,データセットを用いた10分割交差検証を行い,評価指標として適合率,再現率,F値を算出する.不具合発見の予測の正確性を比較するため,「すべてのテストケースで不具合を発見すると予測する」というベースラインを用意し,比較する.

\begin{figure}[H]
    \begin{center}
    \hspace{-15mm}
        \includegraphics[height=200mm]{img/test_method.png}
        \caption{実験の概要図}
        \label{fig:testflow}
    \end{center}
\end{figure}


\subsubsection*{実験目的}
本実験の目的は2つある.
1つ目は,不具合発見とマップ構造の相関を明らかにすることで,提案手法における適用プロセスの妥当性を検証することである.
2つ目は,パラメータ設定から不具合発見を予測するモデルの構築を試み,その予測精度および有効性を評価することである.また,これらの調査に向けて以下の2つのRQを設定した.

\begin{description}
    \item[RQ1:] \rqfirst
    \item[RQ2:] \rqsecond
\end{description}


\subsubsection*{実験設定}
% 本実験では局所ファジングのパラメータに焦点を当てて行う.これは本実験で検出対象として埋め込んだ不具合がスタックであるためである.スタックは主に地形やオブジェクトとの衝突判定によって発生するため,大まかなタスク順序を決定する大域的ファジングよりも,具体的な移動経路や経由地を決定する局所的ファジングの影響度が支配的であると考えられる.

本実験では,不具合発見の要因を分析するために,以下の計4つのパラメータを採用した.これらは,BiFuzzのテストケース生成に使用されるパラメータと,本実験のために独自に追加したマップに関するパラメータに大別される.

BiFuzzにおいて,テストケース(移動経路)の生成に直接関与する以下の3つのパラメータを採用した.
\begin{itemize}
    \item \textbf{cpNum}: 経由地の数
    \item \textbf{cpNum\_range}:経由地の生成範囲
    \item \textbf{cpNum\_dir}: 経由地の生成方向
\end{itemize}
これらを採用した理由は,BiFuzzの局所的ファジングにおいて経路決定に使用されるすべてのパラメータを網羅することで,パラメータが不具合発見に与える影響を漏れなく評価するためである.\\
マップ環境を変化させるためのパラメータとして,本実験独自に以下を採用した.
\begin{itemize}
    \item \textbf{tree}(障害物となる木の数): マップ上にランダムに配置される,衝突判定を持つ木のオブジェクトの総数.
\end{itemize}
これを採用した理由は,マップ構造(障害物の密度)の変化を特徴量として含めることで,RQ1で設定した「マップ構造と不具合発見の相関」を検証するためである.

実験においてパラメータが取りうる値は連続値ではなく離散的な代表値に限定した.これは,全通りの組み合わせを実行することは時間的制約から現実的ではなく,かつパラメータの微小な値の変化は不具合発見の結果に大きく影響しないと考えられるためである.具体的に採用したパラメータとその値の集合を以下に示す.
\begin{itemize}
    \item \textbf{cpNum} : $\{1, 50, 99, 200, 300, 700\}$
    \item \textbf{cpNum\_range} : $\{1, 50, 99\}$
    \item \textbf{cpNum\_dir} : $\{1, 2, 3, 4\}$
    \item \textbf{tree} : $\{0, 500, 1000\}$
\end{itemize}

また,BiFuzzは同一のパラメータ設定であっても実行結果が異なるような非決定性を含んでいる.この影響を考慮し,上記パラメータの組み合わせ1つにつき5回ずつ実行を行い,データを収集することとした.

今回の実験では\figref{fig:stack_bug_object}のようなスタックする不具合を模したオブジェクトをマップ上に100箇所ランダムに配置する.このオブジェクトは外側からかは侵入可能であるが,内側からは脱出不可能になっている.一定時間このオブジェクトの内側にいると「スタックした」と判定される.
\begin{figure}[H]
    \begin{center}
        \includegraphics{img/sample.png}
        \caption{スタックする不具合を模したオブジェクト}
        \label{fig:stack_bug_object}
    \end{center}
\end{figure}

\subsubsection*{対象ゲーム}実験の対象となるビデオゲームとして,BiFuzzの提案論文\cite{Kato2024}でも使用された「STAR COLLECTION」を採用した.本ゲームは ゲームエンジンであるUnity\footnote{\url{https://unity.com/ja}}を用いて開発された,簡易的なオープンワールドゲームである.ゲームの目的は,広大なマップ上に配置された星のアイテムをすべて回収することである.ゲーム内の環境はシンプルに構成されており,マップ上には\figref{fig:player}のようなプレイヤキャラクタ,\figref{fig:star}のような回収対象となる星のアイテム,そして\figref{fig:tree}のような移動の障害物となる木のオブジェクトのみが存在する.

\subsubsection*{分析の枠組み}前節で述べた通り,BiFuzzの実行結果には非決定性があり,同一のパラメータ設定であっても実行ごとに結果が異なる可能性がある.そこで本実験では,構築する予測モデルの堅牢性を多角的に評価するために,収集したデータに対して以下の3つの異なる粒度で正例を定義し,それぞれについて分析を行うこととした.

\begin{enumerate}
    \item \textbf{1回の実行ごとの分析 }:
    5回の試行をそれぞれ独立したデータとして扱い,「その1回の実行で不具合を発見できたか」を予測対象とする.
    
    \item \textbf{5回の実行のうち1回以上不具合を発見 }:
    パラメータの組み合わせごとに集計を行い,5回の試行のうち「少なくとも1回不具合を発見できた」場合を正例とする.

    \item \textbf{5回の実行のうち5回とも不具合を発見 }:
    パラメータの組み合わせごとに集計を行い,5回の試行すべてにおいて不具合を発見できた場合のみを正例とする.
\end{enumerate}

\begin{figure}[H]
    \begin{center}
        \includegraphics{img/sample.png}
        \caption{プレイヤキャラクタ}
        \label{fig:player}
    \end{center}
\end{figure}

\begin{figure}[H]
    \begin{center}
        \includegraphics{img/sample.png}
        \caption{星のオブジェクト}
        \label{fig:star}
    \end{center}
\end{figure}

\begin{figure}[H]
    \begin{center}
        \includegraphics{img/sample.png}
        \caption{木のオブジェクト}
        \label{fig:tree}
    \end{center}
\end{figure}

\subsection{1回の実行で不具合発見するか}\label{chap:test1}
本節では,BiFuzzの1回の実行を1つのデータとし,その実行において不具合が発見されたか否かを予測対象とした場合の結果について述べる.
比較対象として用意した,本設定におけるベースラインの評価結果を表\ref{tab:baseline_result_1}に示す.
\begin{table}[H]
  \centering
  \caption{ベースラインの評価結果 (\testfirst)}
  \label{tab:baseline_result_1}
  \begin{tabular}{l|r} \hline
    評価指標 & 値 \\ \hline \hline
    正解率 & 0.5764 \\
    適合率 & 0.5764 \\
    再現率 & 1.0000 \\
    F値 & 0.7313 \\ \hline
  \end{tabular}
\end{table}
また,今回のモデルの構築に使用したデータセットの分布を\cref{data_set_1}に示す.
\begin{table}[H]
  \centering
  \caption{データセットの分布(\testfirst)}
  \label{data_set_1}
  \begin{tabular}{l|r} \hline
    分類 & 数 \\ \hline \hline
    不具合発見する & 6223\\
    不具合発見しない & 4577\\ \hline
  \end{tabular}
\end{table}


\subsubsection{ロジスティック回帰分析}
ロジスティック回帰分析を用いて導出された,不具合発見確率 $P$ を予測する回帰式を式(\ref{eq:logistic_single_split})に示す.

\begin{equation}
\begin{split}
    \label{eq:logistic_single_split}
    P &= \frac{1}{1 + \exp(-z)} \\
    \text{where, } z &= -1.788594 + 0.000146 \cdot tree + 0.001877 \cdot cpNum \\
      &\quad + 0.032893 \cdot cpNum\_range + 0.080988 \cdot cpNum\_dir
\end{split}
\end{equation}

ここで,各変数の係数に着目すると,$cpNum\_range$や $cpNum\_dir$の係数が正の値となっており,これらの値が増加するほど不具合発見確率が上昇する傾向が見られた.一方で,$tree$の係数は非常に小さく,不具合発見への寄与は限定的であると言える.

表\ref{tab:logistic_single_metrics}に,10分割交差検証によるモデルの評価結果を示す.表\ref{tab:baseline_result_1}より,ベースラインのF値は0.7313であるのに対し,ロジスティック回帰分析を用いた本モデルのF値の平均は0.8397となり,ベースラインを0.1以上上回る結果となった.ベースラインはすべてを陽性と予測するため再現率は1.0000となるが,適合率が0.5764と低い.一方,本モデルは適合率が0.7932とベースラインよりも約0.22高く,不具合発見の可能性が高いケースをより正確に識別できていることがわかる.
\begin{table}[H]
    \caption{ロジスティック回帰分析の評価結果(\testfirst)}
    \label{tab:logistic_single_metrics}
    \centering
    \begin{tabular}{lcccc}
        \hline
        評価指標 & 平均値 & 標準偏差 & 最小値 & 最大値 \\\hline 
        正解率& 0.8036 & 0.0215 & 0.7704 & 0.8574 \\\hline
        適合率 & 0.7932 & 0.0199 & 0.7583 & 0.8431 \\\hline
        再現率 & 0.8922 & 0.0154 & 0.8732 & 0.9244 \\\hline
        F値 & 0.8397 & 0.0169 & 0.8158 & 0.8819 \\\hline
        AUC & 0.8288 & 0.0178 & 0.7994 & 0.8688 \\\hline
    \end{tabular}
\end{table}


\subsubsection{決定木}
決定木を用いてモデルを構築した際の特徴量重要度を表\ref{tab:dt_single_importance}に示す.
なお,本表に示す重要度は,評価指標の算出と同様に10分割交差検証を行い,各試行で得られた重要度の平均値を算出したものである.
結果より,$cpNum\_range$ の重要度が約69\%と最も高く,次いで $cpNum$ が約31\%を占めている.これら2つのパラメータで全体の重要度の99\%以上を占めているため,$tree$ と $cpNum\_dir$ の影響は極めて小さいと言える.
\begin{table}[H]
    \caption{決定木における特徴量重要度(\testfirst)}
    \label{tab:dt_single_importance}
    \centering
    \begin{tabular}{lrrr}\hline
        特徴量 & 平均重要度 & 標準偏差 & 割合(\%) \\\hline
        cpNum\_range & 0.6891 & 0.0035 & 68.91 \\\hline
        cpNum & 0.3083 & 0.0035 & 30.83 \\\hline
        tree & 0.0027 & 0.0004 & 0.27 \\\hline
        cpNum\_dir & 0.0000 & 0.0000 & 0.00 \\\hline
    \end{tabular}
\end{table}

表\ref{tab:dt_single_metrics}に,10分割交差検証による評価結果を示す.F値の平均は0.8631となり,ロジスティック回帰分析と比較して高い予測精度が得られた.
\begin{table}[H]
    \caption{決定木の評価結果(\testfirst)}
    \label{tab:dt_single_metrics}
    \centering
    \begin{tabular}{lcccc}\hline
    評価指標 & 平均値 & 標準偏差 & 最小値 & 最大値 \\\hline
    正解率 & 0.8451 & 0.0162 & 0.8167 & 0.8787 \\\hline
    適合率 & 0.8793 & 0.0158 & 0.8430 & 0.9085 \\\hline
    再現率 & 0.8477 & 0.0160 & 0.8266 & 0.8778 \\\hline
    F値 & 0.8631 & 0.0142 & 0.8403 & 0.8929 \\\hline
    AUC & 0.8648 & 0.0182 & 0.8303 & 0.9034\\\hline
    \end{tabular}
\end{table}

また,表\ref{tab:baseline_result_1}に示したベースラインのF値0.7313と比較すると,決定木モデルのF値の平均は0.8631であり,ベースラインを約0.13上回った.また,適合率に関してもベースラインの適合率0.5764と比較して,本モデルの適合率の平均は0.8793とベースラインよりも約0.30高く,不具合発見の可能性が高いケースをより正確に識別できていることがわかる.

\subsubsection{ランダムフォレスト}
ランダムフォレストを用いた場合の特徴量重要度を表\ref{tab:rf_single_importance}に示す.決定木と同様に,これらの値は10分割交差検証によって得られた各重要度の平均値を示している.
結果として,$cpNum\_range$ が最も高い重要度を示したが,その割合は約74\%と決定木の場合よりもさらに高くなっている.$cpNum$ は約26\%となっているため,決定木と同様にこれら2つの特徴量が支配的で$cpNum\_dir$と$tree$の影響は極めて小さい傾向があると言える.

\begin{table}[H]
    \caption{ランダムフォレストにおける特徴量重要度(\testfirst)}
    \label{tab:rf_single_importance}
    \centering
    \begin{tabular}{lrrr}\hline
        特徴量 & 平均重要度 & 標準偏差 & 割合(\%) \\\hline
        cpNum\_range & 0.7389 & 0.0066 & 73.89 \\\hline
        cpNum & 0.2582 & 0.0064 & 25.82 \\\hline
        tree & 0.0028 & 0.0004 & 0.28 \\\hline
        cpNum\_dir & 0.0000 & 0.0000 & 0.00 \\\hline
    \end{tabular}
\end{table}

表\ref{tab:rf_single_metrics}に,10分割交差検証による評価結果を示す.各評価指標の平均値は決定木の結果と同様の値を示しており,高い精度で不具合発見の予測が可能であることが確認された.
\begin{table}[H]
    \caption{ランダムフォレストの評価結果(\testfirst)}
    \label{tab:rf_single_metrics}
    \centering
    \begin{tabular}{lcccc}\hline
        評価指標 & 平均値 & 標準偏差 & 最小値 & 最大値 \\\hline
        正解率 & 0.8451 & 0.0162 & 0.8167 & 0.8787 \\\hline
        適合率 & 0.8793 & 0.0158 & 0.8430 & 0.9085 \\\hline
        再現率 & 0.8477 & 0.0160 & 0.8266 & 0.8778 \\\hline
        F値 & 0.8631 & 0.0142 & 0.8403 & 0.8929 \\\hline
        AUC & 0.8632 & 0.0171 & 0.8303 & 0.8976 \\\hline
    \end{tabular}
\end{table}

表\ref{tab:baseline_result_1}のベースラインとの比較においても,本モデルのF値の平均は0.8631であり,ベースラインのF値0.7313を大きく上回っている.ランダムフォレストは決定木と同様に0.8793と高い適合率となっており,ベースラインの適合率である0.5764を上回っている.

\subsection{5回の実行のうち1回以上不具合発見するか}\label{chap:test2}
本節では,BiFuzzのパラメータ設定ごとに5回実行を行った際,少なくとも1回以上不具合が発見されたか否かを予測対象とした場合の結果について述べる.
比較対象として用意した,本設定におけるベースラインの評価結果を表\ref{tab:baseline_result_5}に示す.

\begin{table}[H]
  \centering
  \caption{ベースラインの評価結果 (\testsecond)}
  \label{tab:baseline_result_5}
  \begin{tabular}{l|r} \hline
    評価指標 & 値 \\ \hline \hline
    正解率 & 0.8449 \\
    適合率 & 0.8449 \\
    再現率 & 1.0000 \\
    F値 & 0.9159 \\ \hline
  \end{tabular}
\end{table}

また,今回のモデルの構築に使用したデータセットの分布を\cref{data_set_2}に示す.
\begin{table}[H]
  \centering
  \caption{データセットの分布(\testsecond)}
  \label{data_set_2}
  \begin{tabular}{l|r} \hline
    分類 & 数 \\ \hline \hline
    5回実行のうち1回でも不具合発見する & 1824\\
    不具合発見しない & 336\\ \hline
  \end{tabular}
\end{table}

\subsubsection{ロジスティック回帰分析}
ロジスティック回帰分析を用いて導出された,不具合発見確率 $P$ を予測する回帰式を式(\ref{eq:logistic_multi_split})に示す.

\begin{equation}
\begin{split}
    \label{eq:logistic_multi_split}
    P &= \frac{1}{1 + \exp(-z)} \\
    \text{where, } z &= -1.940330 + 0.000237 \cdot tree + 0.002276 \cdot cpNum \\
      &\quad + 0.035500 \cdot cpNum\_range + 0.190446 \cdot cpNum\_dir
\end{split}
\end{equation}

式(\ref{eq:logistic_multi_split})の各変数の係数に着目すると,$cpNum\_dir$の係数が0.190446と最も大きく,次いで$cpNum\_range$が正の値を示している.\cref{chap:test1}の式\ref{eq:logistic_single_split}と比較して,$cpNum\_dir$の影響度が相対的に大きくなっていることが確認できる.

表\ref{tab:logistic_multi_metrics}に,10分割交差検証によるモデルの評価結果を示す.
F値の平均は0.9157であり,表\ref{tab:baseline_result_5}に示したベースラインのF値0.9159とほぼ同等の結果となった.また,再現率が1.0000であることから,本モデルはすべてのテストケースに対して「不具合を発見する」と予測する傾向が強いと言える.

\begin{table}[H]
    \caption{ロジスティック回帰分析の評価結果(\testsecond)}
    \label{tab:logistic_multi_metrics}
    \centering
    \begin{tabular}{lcccc}
        \hline
        評価指標 & 平均値 & 標準偏差 & 最小値 & 最大値 \\\hline 
        正解率 & 0.8444 & 0.0023 & 0.8426 & 0.8472 \\\hline
        適合率 & 0.8444 & 0.0023 & 0.8426 & 0.8472 \\\hline
        再現率 & 1.0000 & 0.0000 & 1.0000 & 1.0000 \\\hline
        F値 & 0.9157 & 0.0013 & 0.9146 & 0.9173 \\\hline
    \end{tabular}
\end{table}

\subsubsection{決定木}
決定木を用いてモデルを構築した際の特徴量重要度を表\ref{tab:dt_multi_importance}に示す.
結果より,$cpNum\_range$ の重要度が約77\%と最も高く,次いで $cpNum$ が約22\%を占めている.これら2つのパラメータで全体の重要度の99\%以上を占めている点は\cref{chap:test1}と同様の傾向である.

\begin{table}[H]
    \caption{決定木における特徴量重要度(\testsecond)}
    \label{tab:dt_multi_importance}
    \centering
    \begin{tabular}{lrrr}\hline
        特徴量 & 平均重要度 & 標準偏差 & 割合(\%) \\\hline
        cpNum\_range & 0.7692 & 0.0119 & 76.92 \\\hline
        cpNum & 0.2238 & 0.0159 & 22.38 \\\hline
        tree & 0.0070 & 0.0073 & 0.70 \\\hline
        cpNum\_dir & 0.0000 & 0.0000 & 0.00 \\\hline
    \end{tabular}
\end{table}

表\ref{tab:dt_multi_metrics}に,10分割交差検証による評価結果を示す.
評価指標の平均値はいずれもロジスティック回帰分析の結果と一致しており,F値は0.9157となった.ベースラインとの比較においても,数値上の差異は見られなかった.

\begin{table}[H]
    \caption{決定木の評価結果(\testsecond)}
    \label{tab:dt_multi_metrics}
    \centering
    \begin{tabular}{lcccc}\hline
    評価指標 & 平均値 & 標準偏差 & 最小値 & 最大値 \\\hline
    正解率 & 0.8444 & 0.0023 & 0.8426 & 0.8472 \\\hline
    適合率 & 0.8444 & 0.0023 & 0.8426 & 0.8472 \\\hline
    再現率 & 1.0000 & 0.0000 & 1.0000 & 1.0000 \\\hline
    F値 & 0.9157 & 0.0013 & 0.9146 & 0.9173 \\\hline
    \end{tabular}
\end{table}


\subsubsection{ランダムフォレスト}
ランダムフォレストを用いた場合の特徴量重要度を表\ref{tab:rf_multi_importance}に示す.
$cpNum\_range$ が約72\%,$cpNum$ が約24\%と,決定木と同様の傾向を示している.

\begin{table}[H]
    \caption{ランダムフォレストにおける特徴量重要度(\testsecond)}
    \label{tab:rf_multi_importance}
    \centering
    \begin{tabular}{lrrr}\hline
        特徴量 & 平均重要度 & 標準偏差 & 割合(\%) \\\hline
        cpNum\_range & 0.7163 & 0.0141 & 71.63 \\\hline
        cpNum & 0.2378 & 0.0142 & 23.78 \\\hline
        tree & 0.0348 & 0.0036 & 3.48 \\\hline
        cpNum\_dir & 0.0111 & 0.0023 & 1.11 \\\hline
    \end{tabular}
\end{table}

表\ref{tab:rf_multi_metrics}に,10分割交差検証による評価結果を示す.
F値の平均は0.9145であり,ロジスティック回帰分析や決定木と同様に,ベースラインのF値0.9159と極めて近い値を示した.再現率は0.9973とわずかに1.0を下回ったものの,ほぼすべてのケースで不具合発見を予測している傾向にある.

\begin{table}[H]
    \caption{ランダムフォレストの評価結果(\testsecond)}
    \label{tab:rf_multi_metrics}
    \centering
    \begin{tabular}{lcccc}\hline
        評価指標 & 平均値 & 標準偏差 & 最小値 & 最大値 \\\hline
        正解率 & 0.8426 & 0.0041 & 0.8333 & 0.8472 \\\hline
        適合率 & 0.8445 & 0.0021 & 0.8426 & 0.8472 \\\hline
        再現率 & 0.9973 & 0.0056 & 0.9835 & 1.0000 \\\hline
        F値 & 0.9145 & 0.0025 & 0.9086 & 0.9173 \\\hline
    \end{tabular}
\end{table}

以上の結果より,本実験設定においては,ロジスティック回帰分析,決定木,ランダムフォレストのいずれの手法を用いても,ベースラインと比較して予測性能の向上は見られなかった.
この要因として,実験データの偏りが挙げられる.本実験で収集したデータにおいて,「5回のうち1回以上不具合を発見する」という条件を満たす正例は全体の約75\%を占めていた.このように正例の割合が極めて高い不均衡なデータセットであったため,学習されたモデルは特徴量に関わらず「不具合発見」と予測する傾向が強まり,その結果,すべてを正と予測するベースライン手法と予測結果に差異が生じなかったと考えられる.


\subsection{5回の実行のうち5回とも不具合発見するか}\label{chap:test3}
本節では,BiFuzzのパラメータ設定ごとに5回実行を行った際,5回の試行すべてにおいて不具合が発見されたか否かを予測対象とした場合の結果について述べる.
比較対象として用意した,本設定におけるベースラインの評価結果を表\ref{tab:baseline_result_all}に示す.

\begin{table}[H]
  \centering
  \caption{ベースラインの評価結果(\testthird)}
  \label{tab:baseline_result_all}
  \begin{tabular}{l|r} \hline
    評価指標 & 値 \\ \hline \hline
    正解率 & 0.3542 \\
    適合率 & 0.3542 \\
    再現率 & 1.0000 \\
    F値 & 0.5231 \\ \hline
  \end{tabular}
\end{table}

また,今回のモデルの構築に使用したデータセットの分布を\cref{data_set_3}に示す.
\begin{table}[H]
  \centering
  \caption{データセットの分布(\testthird)}
  \label{data_set_3}
  \begin{tabular}{l|r} \hline
    分類 & 数 \\ \hline \hline
    5回実行のうち5回全て不具合発見する & 763\\
    不具合発見しない & 1397\\ \hline
  \end{tabular}
\end{table}

\subsubsection{ロジスティック回帰分析}
ロジスティック回帰分析を用いて導出された,不具合発見確率 $P$ を予測する回帰式を式(\ref{eq:logistic_all_split})に示す.

\begin{equation}
\begin{split}
    \label{eq:logistic_all_split}
    P &= \frac{1}{1 + \exp(-z)} \\
    \text{where, } z &= -2.662411 + 0.000315 \cdot tree + 0.002012 \cdot cpNum \\
      &\quad + 0.035818 \cdot cpNum\_range - 0.019180 \cdot cpNum\_dir
\end{split}
\end{equation}

式(\ref{eq:logistic_all_split})の各変数の係数に着目すると,$cpNum\_range$の係数が0.035818と最も大きな正の値を示しており,次いで$cpNum$が正の値となっている.

表\ref{tab:logistic_all_metrics}に,10分割交差検証によるモデルの評価結果を示す.
F値の平均は0.6648であり,表\ref{tab:baseline_result_all}に示したベースラインのF値0.5231と比較して,約0.14上回る結果となった.

\begin{table}[H]
    \caption{ロジスティック回帰分析の評価結果(\testthird)}
    \label{tab:logistic_all_metrics}
    \centering
    \begin{tabular}{lcccc}
        \hline
        評価指標 & 平均値 & 標準偏差 & 最小値 & 最大値 \\\hline 
        正解率 & 0.7306 & 0.0308 & 0.6713 & 0.7778 \\\hline
        適合率 & 0.5949 & 0.0390 & 0.5238 & 0.6591 \\\hline
        再現率 & 0.7561 & 0.0527 & 0.6974 & 0.8701 \\\hline
        F値 & 0.6648 & 0.0346 & 0.6077 & 0.7101 \\\hline
    \end{tabular}
\end{table}

\subsubsection{決定木}
決定木を用いてモデルを構築した際の特徴量重要度を表\ref{tab:dt_all_importance}に示す.
結果より,$cpNum\_range$ の重要度が約65\%,$cpNum$ が約33\%を占めている.\cref{chap:test1}や\cref{chap:test2}と同様に,これら2つのパラメータが不具合発見に対して支配的な影響を持っていることがわかる.

\begin{table}[H]
    \caption{決定木における特徴量重要度(\testthird)}
    \label{tab:dt_all_importance}
    \centering
    \begin{tabular}{lrrr}\hline
        特徴量 & 平均重要度 & 標準偏差 & 割合(\%) \\\hline
        cpNum\_range & 0.6536 & 0.0077 & 65.36 \\\hline
        cpNum & 0.3321 & 0.0069 & 33.21 \\\hline
        tree & 0.0141 & 0.0017 & 1.41 \\\hline
        cpNum\_dir & 0.0003 & 0.0004 & 0.03 \\\hline
    \end{tabular}
\end{table}

表\ref{tab:dt_all_metrics}に,10分割交差検証による評価結果を示す.
F値の平均は0.7769となり,ベースラインのF値0.5231を大きく上回った.また,ロジスティック回帰分析のF値0.6648と比較しても高い予測精度を示している.特に再現率が0.9882と高く,5回とも不具合を発見するパラメータ設定の多くを予測できていることがわかる.

\begin{table}[H]
    \caption{決定木の評価結果(\testthird)}
    \label{tab:dt_all_metrics}
    \centering
    \begin{tabular}{lcccc}\hline
    評価指標 & 平均値 & 標準偏差 & 最小値 & 最大値 \\\hline
    正解率 & 0.7981 & 0.0335 & 0.7500 & 0.8565 \\\hline
    適合率 & 0.6412 & 0.0402 & 0.5846 & 0.7143 \\\hline
    再現率 & 0.9882 & 0.0123 & 0.9610 & 1.0000 \\\hline
    F値 & 0.7769 & 0.0284 & 0.7379 & 0.8287 \\\hline
    \end{tabular}
\end{table}

\subsubsection{ランダムフォレスト}
ランダムフォレストを用いた場合の特徴量重要度を表\ref{tab:rf_all_importance}に示す.
決定木と同様に $cpNum\_range$ と $cpNum$ が高い重要度を示している.

\begin{table}[H]
    \caption{ランダムフォレストにおける特徴量重要度(\testthird)}
    \label{tab:rf_all_importance}
    \centering
    \begin{tabular}{lrrr}\hline
        特徴量 & 平均重要度 & 標準偏差 & 割合(\%) \\\hline
        cpNum\_range & 0.6262 & 0.0064 & 62.62 \\\hline
        cpNum & 0.3153 & 0.0054 & 31.53 \\\hline
        cpNum\_dir & 0.0357 & 0.0012 & 3.57 \\\hline
        tree & 0.0229 & 0.0013 & 2.29 \\\hline
    \end{tabular}
\end{table}

表\ref{tab:rf_all_metrics}に,10分割交差検証による評価結果を示す.
F値の平均は0.7738であり,ベースラインのF値0.5231と比較すると約0.25上昇した.
決定木と同等に高い予測精度が得られた.

\begin{table}[H]
    \caption{ランダムフォレストの評価結果(\testthird)}
    \label{tab:rf_all_metrics}
    \centering
    \begin{tabular}{lcccc}\hline
        評価指標 & 平均値 & 標準偏差 & 最小値 & 最大値 \\\hline
        正解率 & 0.7995 & 0.0299 & 0.7546 & 0.8519 \\\hline
        適合率 & 0.6466 & 0.0373 & 0.5891 & 0.7115 \\\hline
        再現率 & 0.9659 & 0.0257 & 0.9342 & 1.0000 \\\hline
        F値 & 0.7738 & 0.0262 & 0.7396 & 0.8222 \\\hline
    \end{tabular}
\end{table}


\subsection{実験の考察}
以下では,実験結果に基づいてRQ1およびRQ2に対する回答を記述し,考察を行う.

\subsubsection*{RQ1: 不具合発見とマップ構造は関係しているか}

すべての実験設定およびすべての分析手法において,$tree$の特徴量重要度および回帰係数は,他のパラメータと比較して著しく低い値となった.この結果から,本実験で対象としたスタックする不具合に関しては,障害物の数,すなわちマップの構造が変化しても,不具合発見の成否には大きな影響を与えないと考えられる.

不具合発見に支配的な影響を与えているのは,プレイヤキャラクタの移動経路を決定する $cpNum\_range$ および $cpNum$ であった.マップの環境要因よりも,経路の生成方法が重要であると考えられる.


\subsubsection*{RQ2: 提案手法によって不具合発見を予測できる有効なモデルは構築できるか}

本実験では,異なる3つの粒度で予測モデルの構築を試みた.
\cref{chap:test1}においては,決定木およびランダムフォレストを用いることで,ベースラインと比較してF値を約0.13向上させることができた.これは,提案手法によって構築したモデルによって,不具合発見の成否を一定の精度で予測可能であることを示している.

\cref{chap:test2}においては,ベースラインと比較して予測性能の向上は見られなかった.この要因として,実験データの偏りが挙げられる.\cref{data_set_2}より,本実験で収集したデータにおいて,不具合を発見できたテストケースは全体の約84\%を占めていた.このように正例の割合が極めて高い不均衡なデータセットであったため,学習されたモデルは特徴量に関わらず「不具合発見」と予測する傾向が強まり,すべてを正と予測するベースライン手法と差異が生じなかったことに起因すると考えられる.

\cref{chap:test3}においては,決定木およびランダムフォレストが高い再現率を達成した.
これは,再現性高く不具合を発見できるパラメータ設定を,モデルが高い精度で予測できていることを意味する.

以上より,提案手法を用いることで,不具合発見の確率が高いパラメータ設定を予測する有効なモデルが構築できた.
