% \section{実験}
% 本章では,提案手法を対象のゲームに適用し,
% \subsection*{実験設定}


% \subsection*{対象ゲーム: Star Collection}

\section{実験}\label{sec:experiment}
本章では,第\ref{chap:proposal}章で提案した提案手法の有効性を検証するために実施した実験について述べる.
\ref{fig:testflow}に本実験の流れを示す.本実験では,BiFuzzに対して様々なパラメータを適用してテストを実行し,不具合発見の成否を実験結果として記録する.
そして,収集したデータに対し,提案手法の有効性を評価するためのアルゴリズムとして,ロジスティック回帰分析,決定木,ランダムフォレストの3つを採用し,予測モデルの構築を試みる.
構築した各モデルの性能を評価するために,データセットを用いた10分割交差検証を行い,評価指標として適合率,再現率,F値を算出する.また,不具合発見の予測の正確性を比較するため,「すべてのテストケースで不具合を発見すると予測する」というベースラインを用意し,比較する.

\begin{figure}[H]
    \begin{center}
        \includegraphics{img/sample.png}
        \caption{実験の概要図}
        \label{fig:testflow}
    \end{center}
\end{figure}


\subsubsection*{実験目的}
本実験の目的は2つある.
1つ目は,不具合発見とマップ構造の相関を明らかにすることで,提案手法における適用プロセスの妥当性を検証することである.
2つ目は,パラメータ設定から不具合発見を予測するモデルの構築を試み,その予測精度および有効性を評価することである.また,これらの調査に向けて以下の2つのRQを設定した.

\begin{description}
    \item[RQ1:] \rqfirst
    \item[RQ2:] \rqsecond
\end{description}


\subsubsection*{実験設定}
% 本実験では局所ファジングのパラメータに焦点を当てて行う.これは本実験で検出対象として埋め込んだ不具合がスタックであるためである.スタックは主に地形やオブジェクトとの衝突判定によって発生するため,大まかなタスク順序を決定する大域的ファジングよりも,具体的な移動経路や経由地を決定する局所的ファジングの影響度が支配的であると考えられる.

実験においてパラメータが取りうる値は連続値ではなく離散的な代表値に限定した.これは,全通りの組み合わせを実行することは時間的制約から現実的ではなく,かつパラメータの微小な値の変化は不具合発見の結果に大きく影響しないと考えられるためである.具体的に採用したパラメータとその値の集合を以下に示す.
\begin{itemize}
    \item \textbf{cpNum}(経由地数): $\{1, 50, 99, 200, 300, 700\}$
    \item \textbf{cpNum\_range}(経由地の生成範囲): $\{1, 50, 99\}$
    \item \textbf{cpNum\_dir}(経由地の生成方向): $\{1, 2, 3, 4\}$
    \item \textbf{tree}(障害物となる木の数): $\{0, 500, 1000\}$
\end{itemize}

また,BiFuzzは同一のパラメータ設定であっても実行結果が異なるような非決定性を含んでいる.この影響を考慮し,上記パラメータの組み合わせ1つにつき5回ずつ実行を行い,データを収集することとした.

\subsubsection*{対象ゲーム}実験の対象となるビデオゲームとして,BiFuzz の提案論文\cite{Kato2024}でも使用された「STAR COLLECTION」を採用した.本ゲームは ゲームエンジンであるUnity\footnote{\url{https://unity.com/ja}}を用いて開発された,簡易的なオープンワールドゲームである.ゲームの目的は,広大なマップ上に配置された「星のアイテム」をすべて回収することである.ゲーム内の環境はシンプルに構成されており,マップ上にはプレイヤキャラクタ,回収対象となる星のアイテム,そして移動の障害物となる木のオブジェクトのみが存在する.

\subsection{\testfirst}
\subsubsection{ロジスティック回帰分析}
\subsubsection{決定木}
\subsubsection{ランダムフォレスト}

\subsection{\testsecond}
\subsubsection{ロジスティック回帰分析}
\subsubsection{決定木}
\subsubsection{ランダムフォレスト}

\subsection{\testthird}
\subsubsection{ロジスティック回帰分析}
\subsubsection{決定木}
\subsubsection{ランダムフォレスト}

\subsection{実験の考察}