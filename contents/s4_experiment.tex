\section{実験}\label{sec:experiment}
本章では,\chapref{sec:proposal}で提案した提案手法の有効性を検証するために実施した実験について述べる.

\figref{fig:testflow}に本実験の流れを示す.本実験では,BiFuzzに対して様々なパラメータについて5回ずつ適用してテストを実行し,不具合発見の成否を実験結果として記録する.
そして,収集したデータに対し,提案手法の有効性を評価するためのアルゴリズムとして,ロジスティック回帰分析,決定木,ランダムフォレストの3つを採用し,以下の3つの項目に関して予測モデルの構築を試みる.
\begin{itemize}
    \item 1回の実行ごとの分析
    \item 5回の実行のうち1回以上不具合を発見
    \item 5回の実行のうち5回とも不具合を発見
\end{itemize}
構築した各モデルの性能を評価するために,データセットを用いた10分割交差検証を行い,評価指標として適合率,再現率,F値を算出する.不具合発見の予測の正確性を比較するため,「すべてのテストケースで不具合を発見すると予測する」というベースラインを用意し,比較する.

\begin{figure}[H]
    \begin{center}
    \hspace{-15mm}
        \includegraphics[height=200mm]{img/test_method.png}
        \caption{実験の概要図}
        \label{fig:testflow}
    \end{center}
\end{figure}


\subsubsection*{実験目的}
本実験の目的は2つある.
1つ目は,不具合発見とマップ構造の相関を明らかにすることで,提案手法における適用プロセスの妥当性を検証することである.
2つ目は,パラメータ設定から不具合発見を予測するモデルの構築を試み,その予測精度および有効性を評価することである.また,これらの調査に向けて以下の2つのRQを設定した.

\begin{description}
    \item[RQ1:] \rqfirst
    \item[RQ2:] \rqsecond
\end{description}


\subsubsection*{実験設定}
% 本実験では局所ファジングのパラメータに焦点を当てて行う.これは本実験で検出対象として埋め込んだ不具合がスタックであるためである.スタックは主に地形やオブジェクトとの衝突判定によって発生するため,大まかなタスク順序を決定する大域的ファジングよりも,具体的な移動経路や経由地を決定する局所的ファジングの影響度が支配的であると考えられる.

本実験では,不具合発見の要因を分析するために,以下の計4つのパラメータを採用した.これらは,BiFuzzのテストケース生成に使用されるパラメータと,本実験のために独自に追加したマップに関するパラメータに大別される.

BiFuzzにおいて,テストケース(移動経路)の生成に直接関与する以下の3つのパラメータを採用した.
\begin{itemize}
    \item \textbf{cpNum}: 経由地の数
    \item \textbf{cpNum\_range}:経由地の生成範囲
    \item \textbf{cpNum\_dir}: 経由地の生成方向
\end{itemize}
これらを採用した理由は,BiFuzzの局所的ファジングにおいて経路決定に使用されるすべてのパラメータを網羅することで,パラメータが不具合発見に与える影響を漏れなく評価するためである.\\
マップ環境を変化させるためのパラメータとして,本実験独自に以下を採用した.
\begin{itemize}
    \item \textbf{tree}(障害物となる木の数): マップ上にランダムに配置される,衝突判定を持つ木のオブジェクトの総数.
\end{itemize}
これを採用した理由は,マップ構造(障害物の密度)の変化を特徴量として含めることで,RQ1で設定した「マップ構造と不具合発見の相関」を検証するためである.

実験においてパラメータが取りうる値は連続値ではなく離散的な代表値に限定した.これは,全通りの組み合わせを実行することは時間的制約から現実的ではなく,かつパラメータの微小な値の変化は不具合発見の結果に大きく影響しないと考えられるためである.具体的に採用したパラメータとその値の集合を以下に示す.
\begin{itemize}
    \item \textbf{cpNum}(経由地数): $\{1, 50, 99, 200, 300, 700\}$
    \item \textbf{cpNum\_range}(経由地の生成範囲): $\{1, 50, 99\}$
    \item \textbf{cpNum\_dir}(経由地の生成方向): $\{1, 2, 3, 4\}$
    \item \textbf{tree}(障害物となる木の数): $\{0, 500, 1000\}$
\end{itemize}

また,BiFuzzは同一のパラメータ設定であっても実行結果が異なるような非決定性を含んでいる.この影響を考慮し,上記パラメータの組み合わせ1つにつき5回ずつ実行を行い,データを収集することとした.

今回の実験では\figref{fig:stack_bug_object}のようなスタックする不具合を模したオブジェクトをマップ上に100箇所ランダムに配置する.このオブジェクトは外側からかは侵入可能であるが,内側からは脱出不可能になっている.一定時間このオブジェクトの内側にいると「スタックした」と判定される.
\begin{figure}[H]
    \begin{center}
        \includegraphics{img/sample.png}
        \caption{スタックする不具合を模したオブジェクト}
        \label{fig:stack_bug_object}
    \end{center}
\end{figure}

\subsubsection*{対象ゲーム}実験の対象となるビデオゲームとして,BiFuzzの提案論文\cite{Kato2024}でも使用された「STAR COLLECTION」を採用した.本ゲームは ゲームエンジンであるUnity\footnote{\url{https://unity.com/ja}}を用いて開発された,簡易的なオープンワールドゲームである.ゲームの目的は,広大なマップ上に配置された星のアイテムをすべて回収することである.ゲーム内の環境はシンプルに構成されており,マップ上には\figref{fig:player}のようなプレイヤキャラクタ,\figref{fig:star}のような回収対象となる星のアイテム,そして\figref{fig:tree}のような移動の障害物となる木のオブジェクトのみが存在する.

\subsubsection*{分析の枠組み}前節で述べた通り,BiFuzzの実行結果には非決定性があり,同一のパラメータ設定であっても実行ごとに結果が異なる可能性がある.そこで本実験では,構築する予測モデルの堅牢性を多角的に評価するために,収集したデータに対して以下の3つの異なる粒度で正解ラベルを定義し,それぞれについて分析を行うこととした.

\begin{enumerate}
    \item \textbf{1回の実行ごとの分析 }:
    5回の試行をそれぞれ独立したデータとして扱い,「その1回の実行で不具合を発見できたか」を予測対象とする.
    
    \item \textbf{5回の実行のうち1回以上不具合を発見 }:
    パラメータの組み合わせごとに集計を行い,5回の試行のうち「少なくとも1回不具合を発見できた」場合を正とする.これは,BiFuzz のランダム性に左右されず,そのパラメータ設定に「不具合を発見する潜在的な能力があるか」を評価するための基準である.

    \item \textbf{5回の実行のうち5回とも不具合を発見 }:
    パラメータの組み合わせごとに集計を行い,5回の試行すべてにおいて不具合を発見できた場合のみを正とする.これは,そのパラメータ設定が「確実に不具合を発見できるか」という安定性を評価するための基準である.
\end{enumerate}

\begin{figure}[H]
    \begin{center}
        \includegraphics{img/sample.png}
        \caption{プレイヤキャラクタ}
        \label{fig:player}
    \end{center}
\end{figure}

\begin{figure}[H]
    \begin{center}
        \includegraphics{img/sample.png}
        \caption{星のオブジェクト}
        \label{fig:star}
    \end{center}
\end{figure}

\begin{figure}[H]
    \begin{center}
        \includegraphics{img/sample.png}
        \caption{木のオブジェクト}
        \label{fig:tree}
    \end{center}
\end{figure}

\subsection{\testfirst}
\subsubsection{ロジスティック回帰分析}
\subsubsection{決定木}
\subsubsection{ランダムフォレスト}

\subsection{\testsecond}
\subsubsection{ロジスティック回帰分析}
\subsubsection{決定木}
\subsubsection{ランダムフォレスト}

\subsection{\testthird}
\subsubsection{ロジスティック回帰分析}
\subsubsection{決定木}
\subsubsection{ランダムフォレスト}

\subsection{実験の考察}