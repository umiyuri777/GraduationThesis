\begin{abstract} \normalsize
オープンワールド形式のビデオゲームは,他のビデオゲームと比較して状態空間が膨大であり,テストの自動化が困難である.これに対し先行研究では,入力列の変異を大域と局所の2段階で行うことでプレイヤの複雑な動きを再現し,実質的な探索空間を限定してテストを行うBiFuzzというファジングツールが提案されている.

しかし,この手法において,パラメータの決定はテスタに委ねられているという課題がある.パラメータの組み合わせによっては不具合を発見しにくいケースが存在しており,このテストケースを実行する回数を減らしたいが,パラメータと不具合発見率の関係性は明らかになっていない.

そこで本研究では,BiFuzzにおけるパラメータの組み合わせと不具合発見の関係性を調査し,最適なモデルを導出するためのデータ収集および分析プロセスを提案する.提案手法の有効性を検証するため,不具合を再現する一度入ると脱出不可能なオブジェクトをマップ上に配置し,様々なパラメータの組み合わせで各5回のテストを実行・記録した.収集したデータに対し,ロジスティック回帰分析,決定木,ランダムフォレストを用いた分析を行った結果,F値が0.6を超える有効な分類モデルを構築することができた.これにより,提案する分析プロセスが,効率的なテストパラメータの特定に寄与することを確認した.

\end{abstract}
\textbf{キーワード:}ファジング,オープンワールドゲーム,ロジスティック回帰分析,決定木,ランダムフォレスト