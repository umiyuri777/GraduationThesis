\section{提案手法}\label{sec:proposal}
本章では,BiFuzzにおけるパラメータの組み合わせと不具合発見の関係性を調査し,予測モデルを構築して効率的にテストを行うためのテストプロセスについて述べる.

\figref{fig:proposed_method_overview}に提案手法の全体像を示す.提案手法は,過去の実行データから傾向を学習する「モデル構築プロセス」と,構築したモデルを用いて効率的なテスト計画を立てる「適用プロセス」の2つの段階から構成される.次節より,各プロセスの詳細について述べる.


\subsection{モデル構築プロセス}
モデル構築プロセスは,継続的なソフトウェア開発において,実装初期段階に実行されるプロセスである.このプロセスの目的は,BiFuzzの実行結果を収集し,パラメータの組み合わせと不具合発見の成否との関係性を学習した予測モデルを作成することである.本研究ではBiFuzzを拡張し,以下の手順で自動化されたプロセスを実行する.

\begin{enumerate}
    \item \textbf{データ収集}:
    BiFuzzに対して多様なパラメータの組み合わせを設定して実行する.この際,パラメータ空間を網羅的に探索すると状態空間が膨大になり現実的な時間では終了できない.そのため,離散的にサンプリングすることで広範な設定におけるデータを収集し,パラメータと不具合発見の成否の関係性を捉える.この際,BiFuzzにおいて,パラメータはプログラム内の変数として定義されているため,手動による設定変更はコストが高い.そのため,本研究ではBiFuzzを拡張し,パラメータ注入機構を実装した.具体的には,探索対象となるパラメータの組み合わせを列挙したCSV形式の設定ファイルを事前に作成する.そして,BiFuzzの起動時に設定ファイルからデータを読み込み,BiFuzzの内部パラメータ変数に値を注入する.これにより,再コンパイルを行うことなく,リストに従って異なる設定でのテストを連続的に実行することを可能にする.
    
    \item \textbf{実行結果の記録}:
    各パラメータ設定でのファジング実行が終了した後,その実行において不具合発見の成否を記録する.本研究では,実行状態を監視し,結果を自動判定する監視モジュールを実装した.本モジュールは,以下の3つの状態を定義してパラメータセットと紐づけてログファイルに出力する.

    \begin{description}
        \item[不具合発見: ] ゲーム側で定義された異常状態を検知した場合
        \item[タイムアウト: ] 設定された制限時間以内に不具合が検知されなかった場合
        \item[不具合なし: ] ゲームクリア等により正常に終了した場合.
    \end{description}
    これにより,入力であるパラメータと出力である不具合発見の成否が紐づけられたデータセットを作成する.
    
    \item \textbf{モデルの構築}:
    収集したデータセットに対し,回帰分析などの統計的手法や機械学習アルゴリズムを適用することで,パラメータ設定から不具合発見の可能性を予測するモデルを構築する.
\end{enumerate}

\subsection{適用プロセス}
適用プロセスは,継続的なソフトウェア開発において,テスト対象に変更が加えられた際に実行されるプロセスである.開発の進行に伴い,機能追加や修正が行われた場合,再度ファジングによるテストが必要となる.この際,モデル構築プロセスで作成された予測モデルを利用することで,効率的なテスト実行を実現する.手順は以下の通りである.

\begin{enumerate}
    \item \textbf{有効なパラメータの予測}:モデル構築プロセスで作成した予測モデルを用いて,パラメータ設定に対する評価を行う.まず,評価対象となる $N$ 個の候補パラメータセットを用意する.
    ここでの候補選定は,開発者が検証したい値を直接指定することを基本とするが,特に指定がない場合や探索範囲を網羅したい場合は、計算コストを抑えるために一様ランダムに選択するなどの方法を採用する.次に,生成した候補をモデルに入力し,不具合発見の可否を予測する.モデルによって「不具合を発見できる」と分類されたパラメータを,優先的に実行すべき有効なパラメータとして特定する.
    
    \item \textbf{ファジングの実行}:
    予測によって特定された,不具合発見率が高いパラメータ設定を優先的に採用してBiFuzzを実行する.
\end{enumerate}

このように,一度モデルを構築した後は,そのモデルを利用して効果的なパラメータのみを選択して実行することで,限られた時間内であっても効率的に不具合を探索することが可能となる.


\begin{figure}[b]
    \begin{center}
        % \hspace{35mm}
        % \hspace{-35mm}
        \includegraphics[height=150mm]{img/proposed_method_overview_2.png}
        \caption{提案手法の概要図}
        \label{fig:proposed_method_overview}
    \end{center}
\end{figure}