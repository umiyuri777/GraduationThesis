\section{提案手法}\label{chap:proposal}
本章では、BiFuzzにおけるパラメータの組み合わせと不具合発見の関係性を調査し、予測モデルを構築して効率的にテストを行うためのテストプロセスについて述べる.

図\ref{fig:proposed_method_overview}に提案手法の全体像を示す.提案手法は,過去の実行データから傾向を学習する「モデル構築プロセス」と,構築したモデルを用いて効率的なテスト計画を立てる「適用プロセス」の2つの段階から構成される.次節より,各プロセスの詳細について述べる.

\begin{figure}[H]
    \begin{center}
        \hspace{35mm}
        \includegraphics[height=200mm]{img/proposed_method_overview.png}
        \caption{提案手法の概要図}
        \label{fig:proposed_method_overview}
    \end{center}
\end{figure}

\subsection{モデル構築プロセス}
モデル構築プロセスは,ソフトウェアの開発初期段階において実行されるプロセスである.このプロセスの目的は,BiFuzzの実行結果を収集し,パラメータの組み合わせと不具合発見の成否との関係性を学習した予測モデルを作成することである.具体的な手順は以下の通りである.

\begin{enumerate}
    \item \textbf{データ収集}:
    まず,BiFuzzに対して多様なパラメータの組み合わせを設定し,ファジングを実行する.この際,パラメータ空間を網羅的に探索すると状態空間が膨大になり現実的な時間では終了できない.そのため,離散的にサンプリングすることで広範な設定におけるデータを収集し,パラメータと不具合発見の成否の関係性を捉える.
    
    \item \textbf{実行結果の記録}:
    各パラメータ設定でのファジング実行が終了した後,その実行において不具合発見の成否を記録する.これにより,入力であるパラメータと出力である不具合発見の成否が紐づけられたデータセットを作成する.
    
    \item \textbf{モデルの構築}:
    収集したデータセットに対し,回帰分析などの統計的手法や機械学習アルゴリズムを適用することで,パラメータ設定から不具合発見の可能性を予測するモデルを構築する.
\end{enumerate}

\subsection{適用プロセス}
適用プロセスは,継続的なソフトウェア開発において,テスト対象に変更が加えられた際に実行されるプロセスである.開発の進行に伴い,機能追加や修正が行われた場合,再度ファジングによるテストが必要となる.この際,モデル構築プロセスで作成された予測モデルを利用することで,効率的なテスト実行を実現する.手順は以下の通りである.

\begin{enumerate}
    \item \textbf{有効なパラメータの予測}:
    モデル構築プロセスで構築した予測モデルに対し,探索候補となるパラメータの組み合わせを入力する.モデルは,過去の傾向に基づき,各パラメータ設定において不具合が発見できる確率を予測する.これにより,不具合発見しやすいパラメータを特定する.
    
    \item \textbf{ファジングの実行}:
    予測によって特定された,不具合発見率が高いパラメータ設定を優先的に採用してBiFuzzを実行する.
\end{enumerate}

このように,一度モデルを構築した後は,そのモデルを利用して効果的なパラメータのみを選択して実行することで,限られた時間内であっても効率的に不具合を探索することが可能となる.