
\section{提案手法}

本章では,BiFuzzにおけるパラメータ設定を最適化し,効率的な不具合検出を実現するための分析および適用プロセスについて述べる.

\subsection{概要}
前章で述べた通り,BiFuzzは多様なパラメータを持つが,どの組み合わせが不具合発見に寄与するかは明らかになっていない.本研究では,単にパラメータと不具合発見率の関係性を調査するだけでなく,その関係性を利用して継続的な開発におけるテスト効率を向上させる一連のプロセスを提案する.

図\ref{fig:proposed_method_overview}に提案手法の全体像を示す.提案手法は,過去の実行データから傾向を学習する「モデル構築プロセス」と,構築したモデルを用いて効率的なテスト計画を立てる「適用プロセス」の2つの段階から構成される.次節より,各プロセスの詳細について述べる.

%%%%%%%%%%%%%%%画像入れる%%%%%%%%%%%%%%%%%
% \begin{figure}[tb]
%   \centering
%   % 画像ファイル名を指定してください (例: images/proposed_method.png)
%   \includegraphics[width=\linewidth]{images/image_11fcde.jpg}
%   \caption{提案手法の概要}
%   \label{fig:proposed_method_overview}
% \end{figure}

\subsection{モデル構築プロセス}
モデル構築プロセスは,主にソフトウェアの開発初期段階において実行されるプロセスである.このプロセスの目的は,BiFuzzの実行結果をデータとして蓄積し,パラメータの組み合わせと不具合発見の成否との関係性を学習した予測モデルを作成することである.具体的な手順は以下の通りである.

\begin{enumerate}
    \item \textbf{データ収集}:
    まず,BiFuzzに対して多様なパラメータの組み合わせを設定し,ファジングを実行する.この際,パラメータ空間を網羅的,あるいはランダムにサンプリングすることで,広範な設定におけるデータを収集する.
    
    \item \textbf{実行結果の記録}:
    各パラメータ設定でのファジング実行が終了した後,その実行において「不具合を発見できたか否か」を記録する.これにより,入力(パラメータ)と出力(不具合発見の成否)が紐づけられたデータセットを作成する.
    
    \item \textbf{モデルの構築}:
    収集したデータセットに対し,回帰分析などの統計的手法や機械学習アルゴリズムを適用することで,パラメータ設定から不具合発見の可能性を予測するモデルを構築する.なお,本研究における具体的な分析手法や使用したアルゴリズムについては,第5章の実験にて詳述する.
\end{enumerate}

\subsection{適用プロセス}
適用プロセスは,継続的なソフトウェア開発において,テスト対象に変更が加えられた際に実行されるプロセスである.開発の進行に伴い,機能追加や修正が行われた場合,再度ファジングによるテスト(回帰テスト)が必要となる.この際,闇雲にテストを実行するのではなく,モデル構築プロセスで作成された予測モデルを利用することで,効率的なテスト実行を実現する.手順は以下の通りである.

\begin{enumerate}
    \item \textbf{有効なパラメータの予測}:
    構築済みの予測モデルに対し,探索候補となるパラメータの組み合わせを入力する.モデルは,過去の傾向に基づき,各パラメータ設定において不具合が発見できる確率を予測する.これにより,不具合発見の可能性が高い「不具合発見しやすいパラメータ」を特定する.
    
    \item \textbf{ファジングの実行}:
    予測によって特定された,不具合発見率が高いパラメータ設定を優先的に採用してBiFuzzを実行する.
\end{enumerate}

このように,一度モデルを構築した後は,そのモデルを利用して効果的なパラメータのみを選択して実行することで,限られた時間内であっても効率的に不具合を探索することが可能となる.