%% 卒業論文テンプレート
%% 本文以外のファイルは基本的に削除不可

%% 10pt -> 12pt
\documentclass[a4paper, 12pt, dvipdfmx]{jarticle}

%% 全体の設定と表紙の設定ファイル
%% レイアウトに関する設定ファイル,必要に応じて修正してください

%% 以下,使用推奨パッケージ
%% タイトルフォント等の設定
\usepackage{bachelor_thesis}
%% 図の使用
\usepackage[dvipdfmx]{graphicx}

%% 以下は必要に応じて削除しても良い
%% 枠付き文章
\usepackage{ascmac}
%% 特殊文字のエラー回避
\usepackage{textcomp}
%% ソースコードの表示,日本語使用時の文字化け防止
\usepackage[svgnames]{xcolor}
\usepackage{listings,jvlisting,jlisting} 
\usepackage{jlisting}
%% 表のセル縦結合
\usepackage{multirow}
%% 注釈でURLを使う場合のエラー回避
\usepackage{url}
%%他,自由に追加してください
\usepackage{subcaption}
\usepackage{here}
\usepackage{amsmath}
\usepackage{array}
\usepackage{enumitem}
\usepackage{autobreak}
\usepackage{hyperref} 
\usepackage{pxjahyper}
\usepackage{cleveref} 

% 日本語フォーマットの定義
\crefformat{section}{#2第#1節#3}
\crefformat{subsection}{#2第#1節#3}
\crefformat{subsubsection}{#2第#1項#3} % ここを「小節」などに変えてもOK
\crefformat{figure}{#2図#1#3}
\crefformat{table}{#2表#1#3}
\crefformat{equation}{#2式(#1)#3}

\crefname{equation}{式}{式}% {環境名}{単数形}{複数形} \crefで引くときの表示
\crefname{figure}{図}{図}% {環境名}{単数形}{複数形} \crefで引くときの表示
\crefname{table}{表}{表}% {環境名}{単数形}{複数形} \crefで引くときの表示
\crefname{algorithm}{Algorithm}{Algorithm}

\crefname{section}{第}{第}
\creflabelformat{section}{#2#1節#3}
\crefname{subsection}{第}{第}
\creflabelformat{subsection}{#2#1節#3}

%% 余白設定,配布のものより狭くしています
\textheight=20.6truecm                % 高さ
\textwidth=14.5truecm                 % 横幅 (約36文字)
\oddsidemargin=0.6truecm              % 左の空きの幅
\evensidemargin=-3.8truecm            % 右の空きの幅

%% ソースコード表示の設定(参照: https://turgure.hatenablog.com/entry/2016/08/19/183501)
%% ソースコードの文字サイズをより小さくしたい場合は
%% \small部分を\footnotesizeや\scriptsizeにしてください
\lstset{
    frame=single,
    numbers=left,
    tabsize=2,
    columns=fixed,
    basewidth=0.5em,
    basicstyle=\ttfamily\small, 
}



%% 参考文献
\def\thebibliography#1{\section*{参考文献\markboth
 {参 考 文 献}{参 考 文 献}\addcontentsline{toc}{section}{参考文献}}\list
 {[\arabic{enumi}]}{\settowidth\labelwidth{[#1]}\leftmargin\labelwidth
 \advance\leftmargin\labelsep
 \usecounter{enumi}}
% \def\newblock{\hskip .11trueem plus .33trueem minus -.07trueem}
 \def\newblock{\hskip .11em plus .33em minus -.07em}
 \sloppy
 \sfcode`\.=1000\relax}
\let\endthebibliography=\endlist

%% 目次
\makeatletter%%
\renewcommand{\section}{%
\newpage
  \@startsection{section}% #1 見出し
   {1}% #2 見出しのレベル
   {\z@}% #3 横組みの場合,見出し左の空き(インデント量)
   {1.5\Cvs \@plus.5\Cdp \@minus.2\Cdp}% #4 見出し上の空き
   {.5\Cvs \@plus.3\Cdp}% #5 見出し下の空き (負の値なら見出し後の空き)
  {\raggedright\reset@font\large\bfseries}% 左揃え
}%
\makeatother%%

%% タイトル用の情報
\年度{2025}
\学期{秋}
\題目{オープンワールドゲームにおける\\回帰分析を用いた\\ファジング結果の予測手法}
\指導教員{吉田 則裕}
\クラス{AA}
\コース{システムアーキテクト}
\学籍番号{2600220405-2}
\氏名{山口 裕世}


\pagenumbering{roman}

\begin{document}

% マクロ
\newcommand{\rqfirst}{不具合発見とマップ構造は関係しているか}
\newcommand{\rqsecond}{提案手法によって不具合発見を予測できる有効なモデルは構築できるか}


% 表紙 %%%
\maketitle

%% 概要(概要はcontentsフォルダ内のabstruct.texに書く)
\begin{abstract} \normalsize
オープンワールド形式のビデオゲームは,他のビデオゲームと比較して状態空間が膨大であり,テストの自動化が困難である.これに対し先行研究では,入力列の変異を大域と局所の2段階で行うことでプレイヤの複雑な動きを再現し,実質的な探索空間を限定してテストを行うBiFuzzというファジングツールが提案されている.

しかし,この手法において,パラメータの決定はテスタに委ねられているという課題がある.パラメータの組み合わせによっては不具合を発見しにくいケースが存在しており,このテストケースを実行する回数を減らしたいが,パラメータと不具合発見率の関係性は明らかになっていない.

そこで本研究では,BiFuzzにおけるパラメータの組み合わせと不具合発見の関係性を調査し,最適なモデルを導出するためのデータ収集および分析プロセスを提案する.提案手法の有効性を検証するため,不具合を再現する一度入ると脱出不可能なオブジェクトをマップ上に配置し,様々なパラメータの組み合わせで各5回のテストを実行・記録した.収集したデータに対し,ロジスティック回帰分析,決定木,ランダムフォレストを用いた分析を行った結果,F値が0.6を超える有効な分類モデルを構築することができた.これにより,提案する分析プロセスが,効率的なテストパラメータの特定に寄与することを確認した.

\end{abstract}
\textbf{キーワード:}ファジング,オープンワールドゲーム,ロジスティック回帰分析,決定木,ランダムフォレスト

%% 目次の出力(章や節を追加したら自動で反映されます)
\tableofcontents
\clearpage

\pagenumbering{arabic}
%% 以下,本文
%% 章を書いたtexファイルを追加する場合下記に追加していく
%% \input{ファイル名} で新しいファイルをインクルードできます.ファイルは相対パスで指定してください.
%% 追加,削除は自由に行ってください
% 本稿は立命館大学情報理工学部配布の卒論スタイルを利用し, \LaTeX でフォーマットした卒業論文のテンプレートです.
% 主に下記を追記しました.
% \begin{itemize}
% 	\item 目次の追加
% 	\item 余白の変更
% 	\item 本文のサンプル
% 	\item 図表,引用文献のサンプル
% 	\item 参考文献,謝辞の追加  
% \end{itemize}
% 次章より基本文法や図表の挿入について例を示します.
% 必要に応じて本texファイルのソースコードも参照ください.
% 本稿で説明している機能以外は各自でお調べください.

% 新しい章を挿入すると自動で改ページが行われます.

\section{序論}
近年,ビデオゲーム市場は世界規模で拡大の一途をたどっている.一般社団法人コンピュータエンターテインメント協会(CESA)が発行した『CESAゲーム産業レポート2025』によれば,世界市場規模は前年比5.0\%増の31兆円を超え,全世界で市場規模が拡大している\cite{CESA2025}.市場の成長に伴い,開発コストの増加や開発期間の長期化がビデオゲーム開発における深刻な問題点として挙げられている.英国の競争・市場庁(CMA)が2023年に公開した報告書によると,主要なAAAタイトル(大作ゲーム)の開発予算は2億ドルを超える規模に達しており,「Call of Duty」\footnote{https://www.callofduty.com/ja}のような一部のシリーズではすでに3億ドルを超える開発資金が投じられている\cite{CMA2023}.また,開発期間においても長期化の傾向は顕著であり,Unityの2024年ゲーミングレポートによれば,ビデオゲームの発売までの平均期間は,2022年の218日から2023年には304日へと,わずか1年で約40\%も増加している\cite{Unity2024}.

長期化した開発期間の中で,特に大きな割合を占めているのがテスト工程である\cite{ソフトウェア開発データ白書2018-2019}.したがって,開発期間の短縮とコスト削減を実現するためにはこのテスト工程を効率化することが必要であるが,その手法の一つとしてソフトウェアテストの自動化が挙げられる.ビデオゲーム開発においても,テスト自動化を試みた研究は複数存在する\cite{CEDIL2200, CEDIL2209}.しかし,ビデオゲーム開発においては「ゲームが楽しいこと」が最優先事項とされ,その芸術的な側面の強さゆえに,開発の過程で頻繁に要件や仕様が変更されるという特徴がある\cite{Kasurinen2014}.一部のゲーム企業ではスクリプトを用いた自動テスト手法\cite{MurphyHill2014}を採用しているが,要件や仕様の頻繁な変更に対して脆弱であり,変更のたびにテストスクリプトを作り直す必要がある.そのため,現状では自動テストを導入・維持するコストよりも,テスタを雇用して人手によるプレイテスティングを行った方が安価で済むケースが多く,自動化の普及を妨げる要因となっている\cite{MurphyHill2014}.特に,近年主流となっているオープンワールドと呼ばれる,プレイヤが自由にフィールド上を行動できるような形式のビデオゲームでは,テストの自動化はより困難とされる.主な要因として,オープンワールドゲームはプレイヤの自由度が高く,移動可能な範囲やインタラクションの組み合わせが膨大であるため,システムが取りうる状態数が膨大になり,網羅的な探索が難しくなることが挙げられる.これに対し,オープンワールドゲームのテスト自動化手法として,機械学習を用いた手法が提案されている\cite{Fan2022, Wang2023}.例えばMineDojo\cite{Fan2022}では,インターネット上の大規模な動画データと言語記述のペアから学習したMineCLIPという報酬モデルを導入することで,人間のプレイ動画に基づいた人間らしい挙動を学習・再現させ,膨大な状態空間を持つ環境においてもテストを行うことを可能にしている。しかし,この手法ではテストを実行するために大量の学習データを用意する必要があるため,開発段階で導入することは難しいと考えられる.

この課題に対し,オープンワールドゲームのテストにファジングを適用したBiFuzzという手法が提案されている\cite{Kato2024}.ファジング\cite{IPA_Fuzzing}はソフトウェアの脆弱性を検出するソフトウェアテスト手法の一つであり,セキュリティ対策の分野や自動運転システムなど幅広い分野で活用されている\cite{Guo2024}.ファジングは機械学習を用いた手法とは異なり,大量の学習データを必要としないため,開発期間内での導入が容易であるという利点がある.BiFuzzは,プレイヤの操作を入力として扱い,それを変異させることで効率的に不具合を探索するという方法を採用している.

一方で,ファジングには運用上の課題も存在する.ファジングは原理的に入力列を無限に変異させ生成し続けることが可能であるため,実行終了のタイミングを決定することが困難である.特にオープンワールドゲームのように探索空間が膨大な対象においては,単に実行時間を延長したとしても,必ずしも効率的に不具合を発見できるとは限らない.したがって,限られた開発リソースの中で最大限の効果を得るためには,実行結果を予測し,不具合を発見しやすいテストケースを優先的に生成する仕組みが必要である.BiFuzzでは,テストケースの生成に複数のパラメータが用いられる.しかし,現状におけるパラメータの決定はテスタの経験に委ねられており,パラメータの組み合わせと不具合発見率の関係性は明らかになっていない.もし,ファジングの実行結果を事前に予測し,不具合発見に寄与するパラメータを特定できれば,不具合発見の確率が高いテストケースを優先的に実行することが可能となる.その結果,限られた実行時間内であっても最大限の成果が得られ,テスト工程の大幅な効率化が期待できる.

そこで本研究では,BiFuzz におけるパラメータの組み合わせと不具合発見の関係性を調査し,最適なモデルを導出するためのデータ収集および分析プロセスを提案する.具体的には,ファジングの実行結果を収集し,ロジスティック回帰分析,決定木,ランダムフォレストといったアルゴリズムを用いて,「その設定で不具合を発見できるか」を予測するモデルの構築を試みる。本研究の有効性を検証するために,BiFuzzの研究で使用されていたビデオゲームを対象に実験を行い,提案手法によって不具合の発見を予測できるモデルが構築可能であるかを明らかにする。
% \section{準備}
% \subsection{関連技術}
% 基本的な \LaTeX の記法は以下になります.
% より詳細な文法については各自でお調べください.
% \subsubsection*{章,節,項}

% \begin{itemize}
% 	\item \verb|\section{章タイトル}|
% 	\item \verb|\subsection{節タイトル}|
% 	\item \verb|\subsubsection{項タイトル}|
% \end{itemize}

% また,\verb|\subsubsection*{タイトル}|と記述すると,見出しの番号が非表示になります.
% 例えば「2.1.1 章,節,項」が「章,節,項」のみの表示となります.


% \subsubsection*{改行}

% \begin{itemize}
% 	\item \verb|\\|
% 	\item \verb|\par|
% 	\item 空白行
% 	\item \verb|\linebreak|
% \end{itemize}

% タイトル内でも改行できるため,適切な位置で改行を挿入してください.
% なお,改行直後の段落では自動で字下げが行われます.
% 字下げしたくない場合は\verb|\noindent|を段落の直前に挿入してください(本texファイル参照).

% \begin{itembox}[l]{改行の例(ソースコード)}
% 立命館大学 \textbackslash\textbackslash \\
% Ritsumeikan University \textbackslash par \\
% 情報理工学部 \\
% \\
% College of Information Science and Engineering\\
% \\
% \\
% \\
% システムアーキテクトコース \textbackslash linebreak\\
% System Architect Course
% \end{itembox}

% \begin{itembox}[l]{改行の例(出力)}
% 立命館大学 \\
% Ritsumeikan University \par
% 情報理工学部 

% College of Information Science and Engineering



% システムアーキテクトコース \linebreak
% System Architect Course
% \end{itembox}


% \subsubsection*{特殊文字}
% \LaTeX は直接入力ができない文字があります.前項の\verb|\\|もコード中に記入したら改行されてしまい,バックスラッシュが表示されません.
% 他によく使うものだと矢印( $\rightarrow$ , $\leftarrow$ )や 波ダッシュ( $\sim$ ) があります.
% 右向きの矢印は\verb|$\rightarrow$|と入力すると表示されます.
% 波ダッシュは\verb|$\sim$|と記述するか,\verb|\usepackage{textcomp}|などを使うことで記述できます.

% 他,数学記号やギリシャ文字,引用文献の人名におけるウムラウト( \"{a} )やアクセント( \'{o} )なども通常の入力ではエラーが生じる場合があります.
% 必要に応じてコマンドを使用してください.

% \subsubsection*{図}
% 図を挿入する際は,\verb|figure|を用います.
% 図は特に挿入位置を指定する必要は無く,自動で最適な箇所に挿入されます.

% 下記に例を示します.
% 下記は図を挿入しつつ,\verb|\begin{figure}~\end{figure}|で囲むことで図を中央揃えに配置しています.
% 例のように記述した場合,図\ref{fig:samplefig}が表示されます.

% \begin{lstlisting}[language=Tex]
% \begin{figure}[t!]
%     \begin{center}
%        \includegraphic[\linewide]{img/sample.png}
%        \caption{図のサンプル}
%     \end{center}
%     \label{fig:samplefig}
% \end{figure}
% \end{lstlisting}

% \verb|\begin{figure}[t!]|の\verb|[t!]|部分は図の挿入位置です.ソースコードの挿入位置(h),上部(t),下部(b)などがあります.
% 本稿ではレイアウトのため,挿入位置の後に\verb|!|を挿入しています(削除すると図が全て論文末尾に移動します).

% \verb|option|では画像の横幅(width)縦幅(height),倍率(scale)などを指定できます.
% \verb|\linewidth|はページの横幅と同様の値を指します.

% \verb|filename|で挿入する画像のファイル名を指定し,\verb|caption|にキャプションを記入します.
% ファイル名は相対パスで表記してください.

% \begin{figure}[b!]
%     \begin{center}
%         \includegraphics[width=\textwidth]{img/sample.png}
%         \caption{図のサンプル}
%         \label{fig:samplefig}
%     \end{center}
% \end{figure}


% 例では追加で,\verb|\caption{図のサンプル}|と\verb|\label{fig:samplefig}|を用いています.
% \verb|\caption{…}|でキャプションの指定,
% \verb|\label{…}|の使い方は\ref{sec:refexp}項で行います.

% \subsubsection*{表}
% 表の挿入は\verb|table|を用います.
% 下記の例の場合だと表\ref{table:sampletab}が出力されます.
% \begin{lstlisting}[language=Tex]
% \begin{table}[t]
%  \caption{表のサンプル}
%  \begin{center}     
%   \begin{tabular}{c|lll}
%    \hline
%    ID & キャンパス名 & 略称 & 大学  \\ \hline
%    1 & びわこ・くさつキャンパス & BKC & \multirow{2}{*}{立命館大学}\\
%    2 & 大阪・いばらきキャンパス & BKC & \\ 
%    3 & 豊中キャンパス &  & 大阪大学\\
%    \hline
%   \end{tabular}
%  \end{center}
%  \label{table:sampletab}
% \end{table}

% \end{lstlisting}

% \begin{table}[t]
%  \caption{表のサンプル}
%  \label{table:sampletab}
%  \begin{center}     
%   \begin{tabular}{c|l|l|l}
%    \hline
%    ID & キャンパス名 & 略称 & 大学  \\ \hline
%    1 & びわこ・くさつキャンパス & BKC & \multirow{2}{*}{立命館大学}\\ \
%    2 & 大阪・いばらきキャンパス & OIC & \\  \hline
%    3 & \multicolumn{2}{c|}{豊中キャンパス}  & 大阪大学\\
%    \hline
%   \end{tabular}
%  \end{center}
% \end{table}

% \verb|\caption|や\verb|\label|の使い方は\verb|figure|と同じです.

% \subsubsection*{引用文献,注釈}
% 文献の引用には\textbf{bibtex}を用いています.本論文の場合,texファイルに直接引用文献情報を書かず,\verb|reference.bib|というファイルに引用文献を記載しています.
% 引用方法は\ref{sec:ref}項を参照してください.
% bibファイルの一部を以下に示します.

% \begin{lstlisting}[language=Tex]
% @inproceedings{yoon2012,
%     author={YoungSeok Yoon and Brad A. Myers},
%     title={An Exploratory Study of Backtracking Strategies 
%                                                  Used by Developers},
%     booktitle={Proceedings of the 5th International Workshop on 
%                Cooperative and Human Aspects of Software Engineering},
%     year={2012},
%     month={June},
%     pages={138--144}
% }
% \end{lstlisting}

% 引用する対象が会議,ワークショップ等で発表された論文である場合\linebreak
% \verb|@inproceedings|で始め,論文誌の場合\verb|@article|で始めます.
% その後,例でいう\verb|yoon2012|部分は該当の文献を引用する時の名前になります.
% \verb|author|は著者一覧,\verb|title|は論文タイトル,\verb|booktitle|は発表された会議名(正確には該当論文が収録された予稿集名),\verb|year, month|は発表された年月,\verb|page|はページ番号です.
% 論文誌の場合\verb|booktitle|でなく\verb|journal|になります.より詳細な情報はbibtexの書き方など\footnote{https://mathlandscape.com/latex-bib/ など}で調べて下さい.

% 文献以外で説明の補足を行う場合は注釈機能を用います.
% 注釈とはこのように\footnote{https://卒業論文.com/2020/04/16/post-250/}該当のページ内下部に表示されるため,論文を読むときの補足に使われます.
% 記述は,
% スラッシュやチルダ(\verb|〜|)の文字化け回避のため,\verb|setting.tex|に\verb|\usepackage{url}|と記述しています.

% \subsection{関連技術} \label{sec:related}
% Texファイル内の参照を\ref{sec:refexp}項で,参考文献の引用を\ref{sec:ref}項で行います.
% \subsubsection{Tex内の参照} \label{sec:refexp}
% 特定の章や図表などを参照する際,数値を直接入力すると,章や図表を新しく挿入した際,参照番号がずれてしまう恐れがあります.
% そのため,Latexでは\textbf{相互参照}の機能があります.

% 相互参照の流れは,(1)参照したい章や図表に\verb|label|を付与し,
% (2)そのラベルを\verb|ref|コマンドで参照します.

% 例えば,\ref{sec:related}節から\ref{sec:refexp}項までのソースコードは下記のようになっています(わかりやすいように一部修正しています).

% \begin{lstlisting}[language=Tex]
% \subsection{関連技術} 
% Texファイル内の参照を\ref{sec:refexp}項で,
%             参考文献の引用を\ref{sec:ref}項で行います.

% \subsubsection{Tex内の参照} \label{sec:refexp}
% 特定の章や図表などを参照する際,数値を直接入力すると…(略)
% \end{lstlisting}

% 上記をコンパイルすると,下記のように出力されます.

% \begin{lstlisting}[language=Tex]
% 2.2 関連技術
%  Texファイル内の参照を2.2.1項で,参考文献の引用を2.2.2項で行います.

% 2.2.1 Tex内の参照
%  特定の章や図表などを参照する際,数値を直接入力すると…(略)
% \end{lstlisting}

% 相互参照を利用することで,例えば2.2.1項に新しい章が挿入され,現在の2.2.1項が2.2.2項や2.3.1項などに変化したとしても,\verb|``Texファイル内の参照を\ref{sec:refexp}項で''|の部分は常に
% \verb|\label{sec:refexp}|が付与された章を示します.

% 表や図のサンプルでもlabelとrefを利用しているので,本ソースコードをご確認ください.

% \subsubsection{文献の引用} \label{sec:ref}
% 引用のテスト\cite{IT人材白書2020}.
% これ\cite{yoon2012}もこれ\cite{sheil1983}もこれ\cite{sandeep2014}もテスト.
% 複数まとめて記述することもできます\cite{IT人材白書2020, yoon2012,sheil1983,sandeep2014}.
% 現状は引用順に参考文献が並びます.
% アルファベット順に引用文献を並べたいときは,本文末尾付近の\verb|\bibliographystyle{junsrt}|を\verb|\bibliographystyle{jplain}|に変更してください.

% コンパイル時に\verb|[??]|と表示されても,複数回コンパイルを行うと解決されることがあります.

\section{関連技術}
\subsection{オープンワールドゲーム}
オープンワールドゲームとは、広大かつ連続的な探索可能領域を持つビデオゲームのジャンルのことで,広大なマップ上をプレイヤが自由に動き回ることができるという特徴\cite{sca_openworld}がある.個別の場面ごとにマップが分断されることなくシームレスに接続されており,移動にロードを挟まない設計になっている.また,ゲームの進行順序が固定されている従来のゲーム形式と異なり,プレイヤは目的地へ到達するための経路選択や、ゲーム内で提示されるタスクやクエストを消化する順序を自由に決定できる\cite{sca_openworld}.

\subsection{ファジング}
ファジングとは,ソフトウェアテスト手法の一つである\cite{ipa_fuzzing_guide}.ファジングでは初期入力列を繰り返し変異させ,ファズという不具合を引き起こしやすい入力列を生成する.ビデオゲームにおけるファズとは,ゲームを開始した初期状態から不具合を引き起こす状態へ遷移する入力列のことを指す.

\subsection{オープンワールドゲームに対するファジングの適用}
オープンワールドゲームに対するファジングの適用事例として,Katoらによって提案された BiFuzz\cite{Kato2024}が挙げられる.BiFuzzは,オープンワールドゲーム特有の課題である広大な状態空間に対応するため,大域的ファジングと局所的ファジングという2段階でのファジングを行うことが特徴である.
前節で述べた通り,オープンワールドゲームは自由度が高いため,単純なランダム入力では状態空間が爆発的に増大し,網羅的なテストを行うことが困難である.これに対しBiFuzzでは,2段階でのファジングを組み合わせることで,人間らしい動作をする入力列を生成する.このように,プレイヤとしてあり得る挙動に絞って探索を行うことで実質的な状態空間を限定し,従来は困難であったオープンワールドゲームに対するファジングの実行を可能にしている.


\section{提案手法}\label{sec:proposal}
本章では,BiFuzzにおけるパラメータの組み合わせと不具合発見の関係性を調査し,予測モデルを構築して効率的にテストを行うためのテストプロセスについて述べる.

\figref{fig:proposed_method_overview}に提案手法の全体像を示す.提案手法は,過去の実行データから傾向を学習する「モデル構築プロセス」と,構築したモデルを用いて効率的なテスト計画を立てる「適用プロセス」の2つの段階から構成される.次節より,各プロセスの詳細について述べる.


\subsection{モデル構築プロセス}
モデル構築プロセスは,継続的なソフトウェア開発において,実装初期段階に実行されるプロセスである.このプロセスの目的は,BiFuzzの実行結果を収集し,パラメータの組み合わせと不具合発見の成否との関係性を学習した予測モデルを作成することである.本研究ではBiFuzzを拡張し,以下の手順で自動化されたプロセスを実行する.

\begin{enumerate}
    \item \textbf{データ収集}:
    BiFuzzに対して多様なパラメータの組み合わせを設定して実行する.この際,パラメータ空間を網羅的に探索すると状態空間が膨大になり現実的な時間では終了できない.そのため,離散的にサンプリングすることで広範な設定におけるデータを収集し,パラメータと不具合発見の成否の関係性を捉える.この際,BiFuzzにおいて,パラメータはプログラム内の変数として定義されているため,手動による設定変更はコストが高い.そのため,本研究ではBiFuzzを拡張し,パラメータ注入機構を実装した.具体的には,探索対象となるパラメータの組み合わせを列挙したCSV形式の設定ファイルを事前に作成する.そして,BiFuzzの起動時に設定ファイルからデータを読み込み,BiFuzzの内部パラメータ変数に値を注入する.これにより,再コンパイルを行うことなく,リストに従って異なる設定でのテストを連続的に実行することを可能にする.
    
    \item \textbf{実行結果の記録}:
    各パラメータ設定でのファジング実行が終了した後,その実行において不具合発見の成否を記録する.本研究では,実行状態を監視し,結果を自動判定する監視モジュールを実装した.本モジュールは,以下の3つの状態を定義してパラメータセットと紐づけてログファイルに出力する.

    \begin{description}
        \item[不具合発見: ] ゲーム側で定義された異常状態を検知した場合
        \item[タイムアウト: ] 設定された制限時間以内に不具合が検知されなかった場合
        \item[不具合なし: ] ゲームクリア等により正常に終了した場合.
    \end{description}
    これにより,入力であるパラメータと出力である不具合発見の成否が紐づけられたデータセットを作成する.
    
    \item \textbf{モデルの構築}:
    収集したデータセットに対し,回帰分析などの統計的手法や機械学習アルゴリズムを適用することで,パラメータ設定から不具合発見の可能性を予測するモデルを構築する.
\end{enumerate}

\subsection{適用プロセス}
適用プロセスは,継続的なソフトウェア開発において,テスト対象に変更が加えられた際に実行されるプロセスである.開発の進行に伴い,機能追加や修正が行われた場合,再度ファジングによるテストが必要となる.この際,モデル構築プロセスで作成された予測モデルを利用することで,効率的なテスト実行を実現する.手順は以下の通りである.

\begin{enumerate}
    \item \textbf{有効なパラメータの予測}:モデル構築プロセスで作成した予測モデルを用いて,パラメータ設定に対する評価を行う.まず,評価対象となる $N$ 個の候補パラメータセットを用意する.
    ここでの候補選定は,開発者が検証したい値を直接指定することを基本とするが,特に指定がない場合や探索範囲を網羅したい場合は、計算コストを抑えるために一様ランダムに選択するなどの方法を採用する.次に,生成した候補をモデルに入力し,不具合発見の可否を予測する.モデルによって「不具合を発見できる」と分類されたパラメータを,優先的に実行すべき有効なパラメータとして特定する.
    
    \item \textbf{ファジングの実行}:
    予測によって特定された,不具合発見率が高いパラメータ設定を優先的に採用してBiFuzzを実行する.
\end{enumerate}

このように,一度モデルを構築した後は,そのモデルを利用して効果的なパラメータのみを選択して実行することで,限られた時間内であっても効率的に不具合を探索することが可能となる.


\begin{figure}[b]
    \begin{center}
        % \hspace{35mm}
        % \hspace{-35mm}
        \includegraphics[height=150mm]{img/proposed_method_overview_2.png}
        \caption{提案手法の概要図}
        \label{fig:proposed_method_overview}
    \end{center}
\end{figure}
% \section{実験}
% 本章では,提案手法を対象のゲームに適用し,
% \subsection*{実験設定}


% \subsection*{対象ゲーム: Star Collection}

\section{実験}\label{sec:experiment}
本章では,第\ref{chap:proposal}章で提案した分析および適用プロセスの有効性を検証するために実施した実験について述べる.本実験では,実際に BiFuzz を用いてゲームのテストを行い,その実行結果に対してロジスティック回帰,決定木,ランダムフォレストの3つの機械学習アルゴリズムを適用して予測モデルを構築する.構築した各モデルの性能を評価するために,データセットを用いた10分割交差検証(10-fold cross-validation)を行い,評価指標として適合率(Precision),再現率(Recall),F値(F1-score)を算出する.また,本実験における比較の基準(ベースライン)として,「すべてのテストケースで不具合を発見すると予測する(All Positive)」戦略を採用する.これは,予測モデルを用いずに網羅的にテストを実行する従来の運用(Run All)に相当する.本ベースラインの数値を提案手法が上回るかを検証することで,不具合発見の予測の正確性およびテスト効率化の有効性を評価する.
\subsection{実験設定}
本節では,実験におけるパラメータの設定および実行条件について述べる.
\subsubsection{対象パラメータの選定}
BiFuzz には,ゲーム全体の進行順序を決定する大域的ファジング(Global Fuzzing)と,移動経路の多様性を生み出す局所的ファジング(Local Fuzzing)の2種類のファジングが存在する.本研究の目的は,パラメータと不具合発見との関係性を明らかにすることであるが,今回の実験では特に\textbf{局所的ファジングのパラメータ}に焦点を当てて検証を行う.この理由は以下の通りである.第一に,本実験で検出対象として埋め込んだ不具合が「スタック(地形へのはまり)」であるためである.スタックは主に地形やオブジェクトとの微細な衝突判定によって発生するため,大まかなタスク順序を決定する大域的ファジングよりも,具体的な移動経路や経由地を決定する局所的ファジングの影響度が支配的であると考えられる.第二に,BiFuzz の仕様上,大域的ファジングにおけるチェックポイント(タスク位置)は固定されており,移動のバリエーションは主に局所的ファジングによって生成されるためである.
\subsubsection{パラメータ空間の限定}
実験においてパラメータが取りうる値は,連続値ではなく離散的な代表値に限定した.これは,全通りの組み合わせを実行することは時間的制約から現実的ではなく,かつパラメータの微小な値の変化(例:経由地数が1と2の違いなど)は不具合発見の結果に大きく影響しないと考えられるためである.具体的に採用したパラメータとその値の集合を以下に示す.
\begin{itemize}
    \item \textbf{cpNum}(経由地数): $\{1, 50, 99, 200, 300, 700\}$
    \item \textbf{cpNum_range}(経由地の生成範囲): $\{1, 50, 99\}$
    \item \textbf{cpNum_dir}(経由地の生成方向): $\{1, 2, 3, 4\}$
    \item \textbf{tree}(障害物となる木の数): $\{0, 500, 1000\}$
\end{itemize}

\subsubsection{実行回数}
BiFuzz はランダムな要素を含むため,同一のパラメータ設定であっても実行結果が異なる場合がある(非決定性).この影響を考慮し,結果の信頼性を担保するために,上記パラメータの組み合わせ1つにつき5回ずつ実行を行い,データを収集した.
\subsection{対象ゲーム}実験の対象となるビデオゲームとして,BiFuzz の提案論文\cite{Kato2024}でも使用された「STAR COLLECTION」を採用した.STAR COLLECTION は Unity ゲームエンジンを用いて開発された,簡易的なオープンワールドゲームである.ゲームの目的は,広大なマップ上に配置された「星のアイテム」をすべて回収することである.ゲーム内の環境はシンプルに構成されており,マップ上にはプレイヤキャラクタ,回収対象となる星のアイテム,そして移動の障害物となる木のオブジェクトのみが存在する.
\input{contents/s5_conclusion}

%% 謝辞のファイル
%% \addcontentsline{toc}{section}{謝辞}は削除しないでください,目次に表示されなくなります
%% 主査,副査への謝辞は最初に必ず入れてください.他,共同研究や学会,打ち合わせ等でお世話になった方も入れてください.
\section*{謝辞}\addcontentsline{toc}{section}{謝辞}
本研究を進めるにあたり,多くの方々に御指導,御協力、御支援をいただきました.
ここに誠意を添えて御名前を記させていただきます.

立命館大学情報理工学部 システムアーキテクトコース 吉田則裕 教授にはご多忙の中,本研究を進めるにあたり,多くの御意見・御助言をいただきました.心より御礼申し上げます.

立命館大学情報理工学部 システムアーキテクトコース 槇原絵里奈 講師にはご多忙の中,本研究を進めるにあたり,多くの御意見・御助言をいただきました.心より御礼申し上げます.

立命館大学大学院情報理工学研究科 加藤優作 氏には本研究に対し多くの御意見・御助言.深く御礼申し上げます.

自動ソフトウェア工学研究室の研究室の皆様には,研究を通して活発な議論にお付き合いいただきました.感謝を申し上げます.


% \input{contents/sanple}

%% 参考文献の表示(参考文献の詳細はreference.bibに書く)
\bibliographystyle{junsrt}
\bibliography{reference}

%% 付録(付録がない場合はコメントアウトしたまま)
% %% 付録部分,必要に応じてご使用ください
%% \addcontentsline{toc}{section}{付録}は削除しないでください,目次に表示されなくなります
\section*{付録}\addcontentsline{toc}{section}{付録}
\subsection{実験Aのプログラム}
\subsection{実験Bのプログラム}
\subsubsection{実験前:アンケート結果}
\subsubsection{実験後:アンケート結果}


\end{document}
